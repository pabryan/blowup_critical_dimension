\documentclass{amsart}

%\usepackage{etoolbox}
%\makeatletter
%\let\ams@starttoc\@starttoc
%\makeatother
%\makeatletter
%\let\@starttoc\ams@starttoc
%\patchcmd{\@starttoc}{\makeatletter}{\makeatletter\parskip\z@}{}{}
%\makeatother

%\usepackage[parfill]{parskip}

\usepackage[colorlinks=true,linkcolor=blue,citecolor=blue,urlcolor=blue]{hyperref}
\usepackage{bookmark}
\usepackage{amsthm,thmtools,amssymb,amsmath,amscd}

\usepackage[bibstyle=alphabetic,citestyle=alphabetic,backend=bibtex]{biblatex}
\bibliography{Bibliography}

\usepackage{fancyhdr}
\usepackage{esint}

\usepackage{enumerate}

\usepackage{pictexwd,dcpic}

\usepackage{graphicx}

\swapnumbers
\declaretheorem[name=Theorem,numberwithin=section]{thm}
\declaretheorem[name=Remark,style=remark,sibling=thm]{rem}
\declaretheorem[name=Lemma,sibling=thm]{lemma}
\declaretheorem[name=Proposition,sibling=thm]{prop}
\declaretheorem[name=Definition,style=definition,sibling=thm]{defn}
\declaretheorem[name=Corollary,sibling=thm]{cor}
\declaretheorem[name=Assumption,style=remark,sibling=thm]{ass}
\declaretheorem[name=Example,style=remark,sibling=thm]{example}


\numberwithin{equation}{section}

\usepackage{cleveref}
\crefname{lemma}{Lemma}{Lemmata}
\crefname{prop}{Proposition}{Propositions}
\crefname{thm}{Theorem}{Theorems}
\crefname{cor}{Corollary}{Corollaries}
\crefname{defn}{Definition}{Definitions}
\crefname{example}{Example}{Examples}
\crefname{rem}{Remark}{Remarks}
\crefname{ass}{Assumption}{Assumptions}
\crefname{not}{Notation}{Notation}

%Symbols
\renewcommand{\~}{\tilde}
\renewcommand{\-}{\bar}
\newcommand{\bs}{\backslash}
\newcommand{\cn}{\colon}
\newcommand{\sub}{\subset}

\newcommand{\N}{\mathbb{N}}
\newcommand{\R}{\mathbb{R}}
\newcommand{\Z}{\mathbb{Z}}
\renewcommand{\S}{\mathbb{S}}
\renewcommand{\H}{\mathbb{H}}
\newcommand{\C}{\mathbb{C}}
\newcommand{\K}{\mathbb{K}}
\newcommand{\Di}{\mathbb{D}}
\newcommand{\B}{\mathbb{B}}
\newcommand{\8}{\infty}

%Greek letters
\renewcommand{\a}{\alpha}
\renewcommand{\b}{\beta}
\newcommand{\g}{\gamma}
\renewcommand{\d}{\delta}
\newcommand{\e}{\epsilon}
\renewcommand{\k}{\kappa}
\renewcommand{\l}{\lambda}
\renewcommand{\o}{\omega}
\renewcommand{\t}{\theta}
\newcommand{\s}{\sigma}
\newcommand{\p}{\varphi}
\newcommand{\z}{\zeta}
\newcommand{\vt}{\vartheta}
\renewcommand{\O}{\Omega}
\newcommand{\D}{\Delta}
\newcommand{\G}{\Gamma}
\newcommand{\T}{\Theta}
\renewcommand{\L}{\Lambda}

%Mathcal Letters
\newcommand{\cL}{\mathcal{L}}
\newcommand{\cT}{\mathcal{T}}
\newcommand{\cA}{\mathcal{A}}
\newcommand{\cW}{\mathcal{W}}

%Mathematical operators
\newcommand{\INT}{\int_{\O}}
\newcommand{\DINT}{\int_{\d\O}}
\newcommand{\Int}{\int_{-\infty}^{\infty}}
\newcommand{\del}{\partial}

\newcommand{\inpr}[2]{\left\langle #1,#2 \right\rangle}
\newcommand{\fr}[2]{\frac{#1}{#2}}
\newcommand{\x}{\times}
\DeclareMathOperator{\Tr}{Tr}

\DeclareMathOperator{\dive}{div}
\DeclareMathOperator{\id}{id}
\DeclareMathOperator{\pr}{pr}
\DeclareMathOperator{\Diff}{Diff}
\DeclareMathOperator{\supp}{supp}
\DeclareMathOperator{\graph}{graph}
\DeclareMathOperator{\osc}{osc}
\DeclareMathOperator{\const}{const}
\DeclareMathOperator{\dist}{dist}
\DeclareMathOperator{\loc}{loc}
\DeclareMathOperator{\grad}{grad}
\DeclareMathOperator{\ric}{Ric}
\DeclareMathOperator{\Rm}{Rm}
\DeclareMathOperator{\weingarten}{\mathcal{W}}
\DeclareMathOperator{\inj}{inj}

%Environments
\newcommand{\Theo}[3]{\begin{#1}\label{#2} #3 \end{#1}}
\newcommand{\pf}[1]{\begin{proof} #1 \end{proof}}
\newcommand{\eq}[1]{\begin{equation}\begin{alignedat}{2} #1 \end{alignedat}\end{equation}}
\newcommand{\IntEq}[4]{#1&#2#3	 &\quad &\text{in}~#4,}
\newcommand{\BEq}[4]{#1&#2#3	 &\quad &\text{on}~#4}
\newcommand{\br}[1]{\left(#1\right)}

\newcommand{\abs}[1]{\left|{#1}\right|}


%Logical symbols
\newcommand{\Ra}{\Rightarrow}
\newcommand{\ra}{\rightarrow}
\newcommand{\hra}{\hookrightarrow}
\newcommand{\mt}{\mapsto}

%Notes
\newcommand{\pa}[1]{{\color{green} pa: {#1}}}
\newcommand{\jj}[1]{{\color{red} jj: {#1}}}
\newcommand{\mni}[1]{{\color{blue} mni: {#1}}}

%Fonts
\newcommand{\mc}{\mathcal}
\renewcommand{\it}{\textit}
\newcommand{\mrm}{\mathrm}

%Spacing
\newcommand{\hp}{\hphantom}


%\parindent 0 pt

\protected\def\ignorethis#1\endignorethis{}
\let\endignorethis\relax
\def\TOCstop{\addtocontents{toc}{\ignorethis}}
\def\TOCstart{\addtocontents{toc}{\endignorethis}}

\DeclareMathOperator{\E}{\mathcal{E}}
\DeclareMathOperator{\bigo}{\mathcal{O}}
\DeclareMathOperator{\littleo}{o}
\declaretheorem[name=Main Theorem]{mainthm}
\crefname{mainthm}{Main Theorem}{Main Theorems}



\begin{document}

\title[]
 {Blow up of equivariant harmonic map heat flow}

\curraddr{}
\email{}

\dedicatory{}
\subjclass[2010]{}
\keywords{}

\begin{abstract}
\end{abstract}

\maketitle

\section{Introduction}
\label{sec:intro}

We consider the phenomena of \emph{blow up} for solutions of the problem,
\begin{equation}
\label{eq:pde}
\begin{cases}
\partial_t \psi &= \psi_{rr} + \frac{C}{r} \psi_r - \frac{1}{r^2} F(\psi) \\
\psi(r, 0) &= \psi_0(r) \\
\psi(0, t) &= 0, \quad \psi(1, t) = \psi_1, \quad t \in [0, T)
\end{cases}
\end{equation}
where \(C > 0\) is a real constant and \(F\) is an odd \(C^{1,\alpha}\) function on \((-a, a)\) for some \(a \in (0, \infty]\) that is increasing on an interval \((-\delta, \delta)\) around \(0\). Note that \(F(0) = 0\) and \(F'(0) > 0\). When \(a < \infty\) we also assume that \(F(a) = 0\), \(F(z) = A z +  \bigo(z^{1+\alpha})\) as \(z \to a\) and \(\int_0^a F(z) dz = 0\).

Such problems arise in studying equivariant \emph{harmonic map heat flow} and equivariant \emph{Yang-Mills flow}. Let \(u\) solve
\[
\begin{cases}
\frac{1}{2} (u^2)' &= F(u) \\
u(0) &= 0.
\end{cases}
\]
so that \(F(\psi) = u(\psi) u'(\psi)\) and
\[
u(z) = \sqrt{\abs{\int_0^z F(w) dw}}.
\]
The conditions \(F > 0\), \(F(0) = 0\), \(F \in C^{1\,\alpha}\) and \(F'(0) > 0\) ensure that \(u\) is at least \(C^{1,\alpha/2}\).

Define the Riemmanian warped-product manifold \((M, g)\) by
\begin{equation}
\label{eq:warped_product}
M = (0, a) \times \S^1, \quad g = dz^2 + u^2(z) d\theta^2.
\end{equation}

Going the other way, given \(u\) defining \(g\) we recover \(F(\psi) = u(\psi) u'(\psi)\). Either way, equation \eqref{eq:pde}  with \(C = 1\) is precisely the \emph{degree one, equivariant harmonic map heat flow} of maps \(\Psi: (B^2, \delta) \to (M, g)\) where \(\delta\) is the flat metric. Here degree one, equivariant means that \(\Psi\) is of the form, \(\Psi(re^{i\theta}) = (\psi(r), \theta)\). Higher degree maps, i.e. those of the form \(\Psi(re^{i\theta}) = (\psi(r), \ell \theta)\) may also be treated - they simply result in \(F \mapsto \ell^2 F\) and here we absorb the \(\ell^2\) factor into \(F\).

One can also formulate the higher dimensional analogue, with \(g = dz^2 + u^2 g_{\S^{n-1}}\) where \(g_{\S^{n-1}}\) is the round metric on \(\S^{n-1}\). In this case, \(C = n-1\).

Let us also remark that if \(F'(0) = 1\), the metric \(g\) extends to a \(C^{1,\alpha/2}\) metric on the open disc \(D(a)\) of radius \(a\) with \(r = 0\) corresponding to the centre of the disc. Similarly for \(F'(a) = -1\) (for \(a < \infty\)). If both conditions hold, the metric extends to a metric on \(\S^2\). If either condition fails, then we still obtain a metric on \(\S^2\) but with a conical singularity at the point(s) corresponding to \(r=0, a\).

Here are two motivating examples described in \cite{MR2332425}.

\begin{example}
Let \(u = \sin(\psi)\). Then degree one harmonic map heat flow gives \(F = \tfrac{1}{2} \sin(2\psi)\) and \((M, g)\) is the round two-sphere. Degree \(\ell\) harmonic map heat flow gives \(F = \tfrac{\ell^2}{2} \sin(2\psi)\).Higher dimensional, equivariant harmonic maps \(B^n \to \S^n\) with equivariant \(\SO(n-1)\) symmetries gives rise to \(C = n-1\), \(F = \tfrac{n-1}{2} \sin(2\psi)\).
\end{example}

\begin{example}
The \(\SO(4)\) equivariant Yang-Mills flow on \(\R^4\) gives rise to \(C = 1\), \(F = 4 \psi(1 - \psi)(1 - \tfrac{1}{2}\psi)\). The higher dimensional (\(\SO(n)\) equivariant on \(\R^n\)) analogue is \(C = n-3\), \(F = 2(n-2) \psi(1-\psi)(1 - \tfrac{1}{2}\psi)\).
\end{example}

Here we are motivated by generalising some interesting properties of these two examples observed in \cite{MR2332425}. These observations are that in either example, the critical dimension (\(n=2\) for harmonic maps, \(n=4\) for Yang-Mills) in which the problems are conformally invariant gives \(C = 1\) and finite time blow up for the harmonic map heat flows (first proven in \cite{MR1180392}) and no finite time blow up for the Yang-Mills flow (first proven in \cite{MR1600272}). The curious point is that for Yang-Mills flow, the metric \(g\) has a conical singularity but no blow up, whereas for the harmonic map heat flow, the metric is smooth and we do have blow up. Evidently, it is not the conical singularity that determines blow up. The answer given is that the coefficient \(\ell^2 = 1\) for harmonic map heat flow, and \(\ell^2 = 4\) for Yang-Mills flow determines blow up.

For us this translates to letting \(\ell^2 = F'(0)\). In other words, blow up is determined by the \emph{linearisation of \(F\) about \(\psi \equiv 0\)}. Let us define,
\[
\bar{F} (\psi) = F(\psi) - F'(0) \psi
\]
and the linear operator,
\begin{equation}
\label{eq:L}
L(\psi) = \psi_{rr} + \frac{C}{r} \psi_r - \frac{1}{r^2} F'(0) \psi
\end{equation}
so that the harmonic map heat flow, equation \eqref{eq:pde} becomes
\begin{equation}
\partial_t \psi - L(\psi) = - \frac{1}{r^2} \bar{F}(\psi).
\end{equation}

The operator \(L\) is an Euler operator with \emph{characteristic equation}
\begin{equation}
\label{eq:char_eqn}
p(\beta) = \beta^2 + (C-1) \beta - F'(0) = 0.
\end{equation}
The discriminant is
\[
\Delta = (C-1)^2 + 4 F'(0) > 0,
\]
so we always have real, distinct roots
\[
\frac{1 - C}{2} \pm \frac{1}{2} \sqrt{(C-1)^2 + 4 F'(0)}.
\]
Notice that since \(F'(0) > 0\), we have
\[
\frac{1}{2} \sqrt{(C-1)^2 + 4 F'(0)} > \left|\frac{1-C}{2}\right|
\]
and hence one root is positive while the other is negative. Solutions of \(L(\psi) = 0\) are of the form \(\psi(r) = r^{\beta^{\pm}}\) where \(\beta^{\pm}\) are the roots of the characteristic equation \eqref{eq:char_eqn}. The negative root is of no interest since \(r^{\beta_-}\) blows up at \(r = 0\). For the positive root, \(r^{\beta_+}\) equals \(0\) at \(r = 0\) and has finite first derivative precisely when \(r^{\beta_+} \geq 1\).

This behaviour carries over to the flow \eqref{eq:pde} as described by main theorems.

\begin{mainthm}[Finite time blow up for \(C = 1\)]
\label{thm:mainC1}
Let \(\psi\) be a finite energy solution of the harmonic map heat flow equation \eqref{eq:pde} with \(C = 1\). Then \(\psi\) has \(C^1\) blow up in finite time if \(\beta_+ = 1\). If \(\beta_+ > 1\), the boundary condition satisfies \(\psi_1 \in (0, 2a)\) and the initial condition satisfies \(\psi_0(r) = \bigo(r^{\beta_+})\) as \(r\to 0\), then finite time blow up does not occur.
\end{mainthm}

\begin{mainthm}[Finite time blow up for general \(C\)]
\label{thm:mainC}
Let \(\psi\) be a finite energy solution of the harmonic map heat flow equation \eqref{eq:pde} with arbitrary \(C\). {\color{red}Then blow up seems to be determined by \(\beta_+\) and \(\beta_-\) and perhaps also \(\bar{F}\)}.
\end{mainthm}

\begin{rem}
\label{rem:char_eqn_C1}
We recover the results of \cite{MR2332425} by observing that when \(C = 1\), the characteristic equation becomes
\[
p(\beta) = \beta^2 - F'(0)
\]
with roots \(\beta_{\pm} = \pm \sqrt{F'(0)}\). Degree one harmonic map heat flow has \(\beta_+ = 1\) and blow up occurs. More generally, degree \(\ell\) harmonic map heat flow has \(\beta_+ = \ell\) which does not blow up for \(\ell \geq 2\) and finally, Yang-Mills has \(\beta_+ = 2\) once more without blow up.
\end{rem}

Our interpretation is that the harmonic map heat flow, equation \eqref{eq:pde} is the harmonic map heat flow from a flat disc \(D^2\) into a warped product that is also invariant under the action of \(\S^1\) on both source and target. The coefficients \(1/r, 1/r^2\) are singular precisely at the fixed point \(r = 0\) and standard parabolic estimates (see \Cref{lem:apriori_bounds} below) ensure that blow up, if it occurs at all, must occur around this fixed point. Note that the other fixed point of \(\S^2\) (i.e. \(r=a\)) behaves qualitatively different as there is no corresponding fixed point of \(D^2\), but does feature in the theorems through the \(a\) dependence of the \(L_{\infty}\) bounds. The reason the linearised operator \(L\) determines blow up is because \(F'(0)\) dominates as \(\psi \to 0\) which occurs as \(r \to 0\) via the boundary condition \(\psi(0) = 0\); that is via the condition that this point is fixed under the action of \(\S^1\) on both source and target.


The method of proof - as used by many authors \cite{MR2332425,MR1180392} - proceeds by suitable construction of sub and super solutions either forcing or preventing blow up. These are obtained from one-parameter families of stationary solutions \(\xi = \xi_{\lambda}\) satisfying \(L(\xi) = -\tfrac{1}{r^2} \bar{F} (\xi)\) and \(\xi(0) = 0\). Since \(L\) is second order, this leaves the free parameter \(\lambda\) specifying \(\xi'(0)\). By allowing this parameter to depend on \(t\) in just the right way, we obtain the desired sub and super solutions, \(\xi_{\lambda(t)} (r)\). Blow up then occurs (or not) according to the variation of \(\xi\) with respect to \(\lambda\) and whether \(\lambda\) blows up in finite time or not.


\section{Energy and Gradient Flow}
\label{subsec:intro_energygradient}

It is standard that both the harmonic map heat flow and Yang-Mills flow are the gradient flow for Dirichlet energies,
\[
E(\Psi) = \frac{1}{2} \int_M \|\nabla \Psi\|^2 d\mu
\]
where \(\|\cdot\|\) and \(\nabla\) are suitably interpreted. Thus our equation \eqref{eq:pde} should be the gradient flow of energy amongst equivariant solutions. Let us record here the simple statement of this fact and make some basic observations about finite energy solutions.

\begin{defn}
\label{defn:energy}
The energy, \(\E(\psi)\) of a map \(\psi: [0, 1] \to \R\) is defined to be
\[
\E(\psi) = \frac{1}{2} \int_0^1 \left(\psi_r^2 + \frac{1}{r^2} u^2(\psi)\right) r^C dr.
\]
\end{defn}

\begin{lemma}
The equation \eqref{eq:pde} is the negative \(L^2\) gradient flow with respect to the weighted measure \(r^C dr\) for the energy \(\E\) defined in Definition \ref{defn:energy}.
\end{lemma}

\begin{proof}
Let \(\psi(r, t)\) be a compactly supported smooth one-parameter family of smooth maps \(r \mapsto \psi(r, t)\).

Then we compute
\[
\begin{split}
\partial_t \E(\psi) &= \partial_t \frac{1}{2} \int_0^1  \left(\psi_r^2 + \frac{1}{r^2} u^2(\psi)\right) r^C dr \\
&= \int_0^1  \left(\psi_r \psi_{rt} + \frac{1}{r^2} u(\psi)u'(\psi) \psi_t \right) r^C dr \\
&= \int_0^1 \left((\psi_r \psi_t r^C)_r - \psi_{rr} \psi_t r^C - C r^{C-1} \psi_r \psi_t + \frac{1}{r^2} F(\psi) \psi_t r^C\right) dr \\
&= - \int_0^1 \left(\psi_{rr} + \frac{C}{r} \psi_r - \frac{1}{r^2} F(\psi)\right) \psi_t r^C dr.
\end{split}
\]
\end{proof}

\begin{rem}
\label{rem:finite_energy}

Finite energy functions \(\psi\) must satisfy
\[
\int_0^1 \psi_r^2 r^C dr < \infty \text{ and } \int_0^1 u^2(\psi(r)) r^{C-2} dr < \infty
\]
Then, as \(r\to 0\), \(\psi_r\) must blow up slower than \(r^{-(1+C)/2}\) so that \(\psi = \littleo(r^{(1-C)/2})\).

Also, the assumptions on \(F\) imply \(F(z) = F'(0) z + \bigo(z^{1+\alpha})\) and hence
\[
u^2(z) = \int F(z) dz = \frac{F'(0))}{2} z^2 + \bigo(z^{2+\alpha}).
\]
Thus finite energy solutions satisfy for any \(\rho \in (0, 1)\),
\[
\int_0^{\rho} \psi(r)^2 r^{C-2} dr < \infty
\]
and hence
\[
\psi(r) = \littleo\left(r^{\tfrac{1-C}{2}}\right)
\]
as \(r \to 0\). In particular, for \(C = 1\).
\[
\psi(r) = \littleo(1)
\]
as \(r \to 0\).
\end{rem}

\section{Stationary solutions}
\label{sec:stationary}

In this section we study stationary solutions with suitable boundary data with a view to creating sub and super solutions in the next section. Our stationary solutions satisfy
\begin{equation}
\label{eq:stationary}
L(\xi) = \xi_{rr} + \frac{C}{r} \xi_r - \frac{F'(0)}{r^2} \xi = \frac{1}{r^2} \bar{F}(\xi).
\end{equation}
Writing \(u = \xi, v = \xi_r\) this is equivalent to the system
\[
\begin{pmatrix}
u' \\
v'
\end{pmatrix}
=
\begin{pmatrix}
v \\
\frac{1}{r^2} F(u) - \frac{C}{r} v
\end{pmatrix}.
\]
For each fixed \(r > 0\), the right hand side is a Lipschitz function of \((u,v)\) since \(F\) is Lipschitz, but it is not uniformly Lipschitz in \(r\). The right hand side is also continuous with respect to \(r\) for \(r > 0\) but only measureable for \(r \in [0, 1]\). The Cartheodary theorem \emph{does not} guarantee the existence of an absolutely continuous solution \((u, v)\) on \([0, \rho_1)\) for some  \(\rho_1 > 0\) since the coefficient functions \(1/r\), \(1/r^2\) while measureable, are not integrable. However, below we obtain solutions by explicitly solving the equation.

On the other hand, if we can obtain a solution \(\xi\) on \([0,\rho_1)\) for some \(\rho_1 > 0\), then this solution may be extended uniquely to \(r \in [0, 1]\) by choosing any \(\rho_2 \in (0, \rho_1)\) and solving the uniformly elliptic problem \eqref{eq:stationary} subject to the boundary conditions \(\psi(\rho_2) = \xi(\rho_2)\), \(\psi(1) = \psi_1\) on \([\rho_1, 1]\). Recall \(\psi_1\) is any real number occurring as the boundary data at \(r=1\) in \eqref{eq:pde}. The solution so obtained is uniformly \(C^{2,\alpha}\) on any interval \([\rho_2, 1]\) where \(\rho_2 > 0\) but not generally up to all of \([0, 1]\).

Finally, note that requiring finite energy solutions and imposing the initial condition \(\psi(0) = 0\) will provide us with a one-parameter family of solutions, parametrised by \(\psi'(0)\) with which to construct sub and super solutions for the blow up argument later.

\subsection{Stationary solutions for \(C=1\)}

Let us first consider the case \(C = 1\) which is satisfied in critical dimension. It will be convenient to work directly with \(F\) rather than the linearisation.

\begin{lemma}
\label{lem:stationaryC1}
For each \(\lambda > 0 \) there exists a \emph{strictly monotone} stationary solution \(\xi_{\lambda}\) (solving \eqref{eq:stationary}) with the following properties:
\begin{enumerate}
\item \(\xi_{\lambda} (r) = \lambda r^{\sqrt{F'(0)}} + \bigo(r^{(1+\bar{\alpha}) \sqrt{F'(0)}}) \text{ as } r \to 0\) for every \(\bar{\alpha} \in (0, \alpha)\), \label{itm:stationaryC1_asymptotic}
\item For every \(r > 0\), the pointwise limit, \(\lim_{\lambda \to 0} \xi_{\lambda} (r) = 0\), \label{itm:stationaryC1_lambda_0}
\item For every \(r > 0\), the pointwise limit, \(\lim_{\lambda \to \infty} \xi_{\lambda} (r) =  a \). \label{itm:stationaryC1_lambda_infty}
\end{enumerate}
\end{lemma}

\begin{proof}
We can solve equation \eqref{eq:stationary} explicitly (in terms of \(F\)) by changing variables, \(y = e^w\) and letting \(\zeta(w) = \xi(e^w)\), whence our problem becomes
\[
\begin{cases}
\zeta'' &= e^{2w} \xi'' + e^w \xi' = F (\zeta) \\
\zeta(-\infty) &= 0.
\end{cases}
\]
Multiplying by \(\zeta'\) this becomes
\[
[(\zeta')^2]' = 2 F (\zeta) \zeta'.
\]
Integrating gives
\[
(\zeta')^2 = G(\zeta) + C_1, \quad \text{where} \quad G'(Z) = 2 F(Z).
\]
By adjusting \(C_1\) if necessary, we may assume that \(G(0) = 0\). Then, referring to definition \ref{defn:energy} and remark \ref{rem:finite_energy}, in order to a obtain finite energy solution we require that
\[
\frac{1}{2} \int_0^1 \left(\psi_r^2 + \frac{1}{r^2} u^2(\psi)\right) r dr < \infty.
\]
For the gradient term, substituting
\[
\psi(r) = \zeta(\ln r)
\]
with \((\zeta')^2 = G(\zeta) + C_1\) we get
\[
\frac{1}{2} \int_0^1 \psi_r^2 r dr  = \frac{1}{2} \int_0^1 \frac{G(\zeta(\ln r^)) + C_1}{r} dr < \infty.
\]
Now if \(C_1 \ne 0\), then since \(\zeta(-\infty) = 0\) and \(G(0) = 0\), by continuity there is an interval \([0, \rho]\) on which \(\abs{G(\zeta(\ln r))} < \abs{C_1}/2\) and consequently
\[
\frac{\abs{G(\zeta(\ln r)) + C_1}}{r} dr \geq \frac{\abs{C_1}}{2} \frac{1}{r}
\]
which is not integrable. Therefore we conclude that \(C_1 = 0\) and in fact,
\begin{equation}
\label{eq:G}
G(z) = 2 \int_0^z F (s) ds = \int_0^z (u^2)' (s) ds = u^2 (z).
\end{equation}
In particular, \(G > 0\) on \((0,a)\) and \(G(0) = G(a) = 0\).

Thus we must solve
\begin{equation}
\label{eq:dzeta}
\frac{\zeta'}{\sqrt{\abs{G(\zeta)}}} = 1
\end{equation}
Notice that near \(z = 0\) we have \(F (z) = F'(0) z + \bigo(z^{1+\alpha})\) so that
\[
G'(z) = 2F(z) \Rightarrow G(z) = z^2F'(0) + \bigo(z^{2 + \alpha})
\]
and so
\[
\sqrt{\abs{G(\zeta)}} = z \sqrt{F'(0) + \bigo(z^{\alpha})} = \sqrt{F'(0)} z + \bigo(z^{1 + \alpha})
\]
is continuous (even differentiable) near \(z = 0\). In fact, explicitly,
\[
\frac{1}{\sqrt{\abs{G(z)}}} = \frac{1}{u(z)}
\]
is as regular as \(u\) is away from \(0, a\).

Now our solution satisfies
\[
H (\zeta(w)) = w + \mu, \quad \text{where} \quad H'(Z) = \frac{1}{\sqrt{\abs{G(Z)}}}.
\]
for an arbitrary constant \(\mu\). Explicitly, choosing any \(a_0 \in (0, a)\) we have
\begin{equation}
\label{eq:H}
H(z) = \int_{a_0}^z \frac{1}{u(s)} ds.
\end{equation}
Varying \(a_0\) just has the effect of varying \(\mu\).

Note that since \(u(z) = \sqrt{F'(0)} z + \bigo(z^{1 + \alpha})\) near \(z=0\) and near \(z=a\), we find that \(1/u\) is not summable near \(z = 0, a\) and hence
\[
H(0) = - \int_0^{a_0} \frac{1}{u(s)} ds = - \infty, \quad H(a) = \int_{a_0}^a \frac{1}{u(s)} ds = \infty.
\]
This is also why we cannot define \(H(z) = \int_{0}^z \frac{1}{u(s)} ds\). More importantly, we have
\[
H'(z) = \frac{1}{u(z)} > 0, \quad z \in (0, a).
\]
Thus \(H\) is strictly increasing and hence a \(C^1\)-diffeomorphism of \((0, a)\) onto \(\R\) so that \(H^{-1}\) is defined and \(C^1\).

In terms of \(\xi(r) = \xi(r) = \zeta(\ln r)\), we have the representation formula
\begin{equation}
\label{eq:stationaryC1}
\xi(r) = H^{-1} (\ln r + \mu) = H^{-1} (\ln \lambda r)
\end{equation}
where \(\lambda = e^{\mu}\). Thus since \(H^{-1}\) maps \(\R\) into \((0, a)\) we have \(0 < \xi(r) < a\) for all \(r \in (0, 1)\) and so our solutions automatically have \(L^{\infty}\) bounds independent of \(\lambda\). We may also use the implicit relation
\[
H (\xi(r)) =  \ln \lambda r.
\]

Either way, we derive:
\begin{align*}
\xi'(r) &= \frac{1}{r}  \sqrt{G(\xi(r))} \\
\xi''(r) &= \frac{1}{r} \frac{G'(\xi(r)) \xi'(r)}{2 \sqrt{G(\xi(r))}} - \frac{1}{r^2} \sqrt{G(\xi(r))} \\
&= \frac{1}{r^2} \left(F(\xi(r))- \sqrt{G(\xi(r))}\right).
\end{align*}
from which we easily verify that \(\xi\) is indeed a solution to equation \eqref{eq:stationary}.

Now we just verify the properties asserted in the statement of the lemma. First, we already observed that \(H^{-1}\) is strictly monotone increasing with \(H^{-1}(-\infty) = 0\) and \(H^{-1}(\infty) = a\). Thus since \(\ln \lambda r\) is also strictly monotone increasing we get that \(\xi = H^{-1} (\ln \lambda r)\) is strictly monotone increasing. We also obtain the last two properties:

Property (\ref{itm:stationaryC1_lambda_0}) follows since for any \(r > 0\),
\[
\lim_{\lambda \to 0} H^{-1}(\ln \lambda r) = H^{-1}(-\infty) = 0.
\]
Property (\ref{itm:stationaryC1_lambda_infty}) follows since for any \(r > 0\),
\[
\lim_{\lambda \to \infty} H^{-1}(\ln \lambda r) = H^{-1}(\infty) = a.
\]

Finally, to verify property (\ref{itm:stationaryC1_asymptotic}), first observe that replacing \(F\) by
\[
\tilde{F} = \frac{1}{F'(0)} F
\]
we have \(\tilde{F}'(0) = 1\). Let us also write \(\tilde{H}^{-1}\) for the corresponding \(H\) in which case
\[
\tilde{\xi}(s) = \tilde{H}^{-1} (\ln \lambda s)
\]
satisfies
\[
\tilde{\xi}_{ss} + \frac{1}{s} \tilde{\xi}_s = \frac{1}{s^2} \tilde{F}(\tilde{\xi}).
\]
Then in fact \(\xi(r) = \tilde{\xi}(r^{\sqrt{F'(0)}})\) since letting \(s = r^{\sqrt{F'(0)}}\) and \(\ell = \sqrt{F'(0)}\), we have
\[
\begin{split}
\xi_{rr} + \frac{1}{r} \xi_r &= \ell^2 r^{2(\ell - 1)} \tilde{\xi}_{ss} + \ell (\ell - 1) r^{\ell - 2} \tilde{\xi}_s + \frac{1}{r} \ell r^{\ell - 1} \tilde{\xi}_s \\
&= \ell^2 r^{-2} \left((r^{\ell})^2\tilde{\xi}_{ss} + r^{\ell} \tilde{\xi}_s\right) = \ell^2 r^{-2} \left(s^2 \tilde{\xi}_{ss} + s \tilde{\xi}_s\right) \\
&= \frac{1}{r^2} F'(0) \tilde{F}(\tilde{\xi}(r^{\ell})) = \frac{1}{r^2} F(\xi).
\end{split}
\]

Thus in fact it suffices to show
\[
\xi_{\lambda} (r) = \lambda r + \bigo(r^{1 + \alpha}) \text{ as } r \to 0
\]
under the assumption \(F'(0) = 1\) (and also relabelling \(\lambda \mapsto \lambda^{1/\sqrt{F'(0)}}\) which doesn't affect the other required properties).

To this end, note that
\[
H(z) = \int \frac{1}{\sqrt{|G(s)|}} ds = \int \frac{ds}{s + \bigo(s^{1+\alpha})} = \int \frac{1}{s}(1 + \bigo(s^{\alpha})) ds = \ln z + \bigo(z^{\alpha})
\]
so that for any \(\epsilon > 0\), there is a \(\delta \in (0, 1)\) such that for \(0 < z < \delta\),
\begin{equation}
\label{eq:H_first_order_ln}
H(z) \geq (1 + \epsilon) \ln z
\end{equation}
for otherwise there exists a sequence \((z_n)\) with \(0 < z_n < 1/n\) such that
\[
H(z_n) - \ln z_n < \epsilon \ln z_n \to -\infty
\]
as \(n \to \infty\). But this contradicts \(H(z) = \ln z + \bigo(z^{\alpha})\) which implies that there is a \(B > 0\), such that for \(n\) sufficiently large,
\[
\abs{H(z_n) - \ln z_n} \leq B z_n^{\alpha} \to 0
\]
as \(n \to \infty\). Note that the restriction \(\delta < 1\) is just to ensure that \(\ln z < 0\) so we don't need to bother with absolute values in equation \eqref{eq:H_first_order_ln}.

Then since \(H^{-1}(-\infty) = 0\) and \(H^{-1}\) is monotonically increasing, for \(w \in (-\infty, H(\delta))\) we have \(H^{-1}(w) \in (0, \delta)\) and hence equation \eqref{eq:H_first_order_ln} implies
\[
w = H(H^{-1} (w)) = \geq (1 + \epsilon) \ln H^{-1}(w).
\]
Taking the exponential,
\begin{equation}
\label{eq:Hinv_first_order_exp}
e^w \geq H^{-1}(w)^{1+\epsilon}
\end{equation}
for \(w \in (-\infty, H(\delta))\).

On the other hand, again since \(H(z) = \ln z + \bigo(z^{\alpha})\),
\begin{equation}
\label{eq:w_asymptotics}
\begin{split}
\abs{H^{-1}(w) - e^w} &= \abs{z - e^{H(z)}} = \abs{z - e^{\ln z + \bigo(z^{\alpha})}} \\
&= \abs{z - e^{\bigo(z^{\alpha})} z} = \abs{(1 - e^{\bigo(z^{\alpha})})z} \\
&\leq A z^{1+\alpha} \\
&= A (H^{-1}(w))^{1+\alpha}
\end{split}
\end{equation}
for some \(A > 0\) and \(z \in [0, \eta)\) for some \(\eta > 0\), or equivalently for \(w \in (-\infty, H(\eta))\).

Combining equations \eqref{eq:Hinv_first_order_exp} and \eqref{eq:w_asymptotics} we find that for \(w \in (-\infty, M_{\epsilon})\) where \(M_{\epsilon} = \min\{H(\delta(\epsilon)), H(\eta)\}) > - \infty\), we have
\begin{equation}
\label{eq:Hinv_first_order_exp_estimate}
\abs{H^{-1}(w) - e^w} \leq A (e^w)^{(1+\alpha)/(1+\epsilon)} \leq A e^{(1+\bar{\alpha}) w}
\end{equation}
where
\[
\bar{\alpha} = \frac{\alpha - \epsilon}{1 + \epsilon}
\]
ranges over all values in \((0, \alpha)\) as \(\epsilon\) ranges over \((0, \alpha)\). To finish, let \(w = \ln \lambda r\) in equation \eqref{eq:Hinv_first_order_exp_estimate} to obtain that for \(r \in (0, e^{M_{\epsilon}}/\lambda)\),
\[
\begin{split}
\abs{\xi(r) - \lambda r} &= \abs{H^{-1} (\ln \lambda r) - e^{\ln \lambda r}} \leq A (e^{\ln \lambda r})^{(1 + \bar{\alpha})} = A\lambda^{1+\bar{\alpha}} r^{1+\bar{\alpha}}
\end{split}
\]
as required.
\end{proof}

\begin{example}[Harmonic map heat flow in \(\S^2\)]

Here we have
\[
F(Z) = \frac{\ell^2}{2} \sin(2Z)
\]
so that
\[
G(Z) = u(Z) = \ell^2 \sin^2(Z)
\]
and
\[
H(Z) = -\frac{1}{2\ell} \ln \left(\frac{1+\cos(Z)}{1-\cos(Z)}\right) = \frac{1}{\ell}\ln \left(\tan(Z/2)\right)
\]
with inverse
\[
H^{-1}(W) = \arccos\left(\frac{e^{-2\ell W} - 1}{e^{-2\ell W} + 1}\right) = 2 \arctan(e^{\ell W})
\]
Thus solution is
\[
\xi(r) = H^{-1} (\ln \lambda r) = \arccos\left(\frac{(\lambda r)^{-2\ell} - 1}{(\lambda r)^{-2\ell} + 1}\right) = 2\arctan((\lambda r)^{\ell}).
\]

Up to relabeling \(\lambda^{-\ell} \mapsto \lambda\) this is the function used in \cite{MR1180392} and \cite{MR2332425}.
\end{example}

\subsection{Stationary solutions for arbitrary \(C\)}

In this case, we don't have the same integrability structure as with \(C = 1\). However, we can still solve (somewhat less explicitly) by factoring \(L\) as the composition of two first order differential operators.

\begin{lemma}
\label{lem:stationaryC}
For each \(\lambda > 0 \) there exists a \emph{strictly monotone} stationary solution \(\xi_{\lambda}\) (solving \eqref{eq:stationary}) with the following properties:
\begin{enumerate}
\item \(\xi_{\lambda} (r) = \sqrt{F'(0)} \lambda r^{\beta_+} + \littleo(r^{\beta_+}) \text{ as } r \to 0\) where \(\beta_+\) is the positive root of the characteristic equation, \label{itm:stationaryC_asymptotic}
\item For every \(r > 0\), the pointwise limit, \(\lim_{\lambda \to 0} \xi_{\lambda} (r) = 0\), \label{itm:stationaryC_lambda_0}
\item For every \(r > 0\), the pointwise limit, \(\lim_{\lambda \to \infty} \xi_{\lambda} (r) = a\). \label{itm:stationaryC_lambda_infty}
\end{enumerate}
\end{lemma}

\begin{proof}
We define
\begin{align*}
L_+ &= \partial_r - \frac{\beta^+ - 1}{r} = r^{\beta^+ - 1} \partial_r (r^{-(\beta^+ - 1)} \cdot) , \\
L_- &= \partial_r - \frac{\beta^-}{r} = r^{\beta^-} \partial_r (r^{-\beta^-} \cdot).
\end{align*}
Since \(\beta^{\pm}\) are the roots of the characteristic equation,
\[
\beta^2 + (C - 1) \beta - F'(0) = 0
\]
we have
\[
- \beta^+ - \beta^- = C-1, \quad \beta^+ \beta^- = -F'(0).
\]
Therefore,
\[
L_+ \circ L_- = \partial_r^2 + \frac{-(\beta^+ - 1) -\beta^-}{r} {\partial_r} + \frac{\beta^+ \beta^-}{r^2} = L.
\]

Then our stationary solution satisfies
\[
L_+ \circ L_- \xi = \frac{1}{r^2} \bar{F}(\xi)
\]
or equivalently
\[
\xi = L_-^{-1} \circ L_+^{-1} \left(\frac{1}{r^2} \bar{F}(\xi)\right).
\]
where \(L_-^{-1}, L_+^{-1}\) are any right inverses. Note that \(L_{\pm}\) being first order, linear differential operators have non-unique right inverses, determined up to arbitrary constants \(C_{\pm}\). Explicitly,
\begin{align*}
L_+^{-1} (\varphi) &= r^{\beta^+ - 1} \int r^{-(\beta^+ - 1)} \varphi(r) dr + C_+ r^{\beta^+-1}, \\
L_-^{-1} (\varphi) &= r^{\beta^-} \int r^{-\beta^-} \varphi(r) dr + C_- r^{\beta^-}
\end{align*}
and hence
\[
\begin{split}
\xi &= r^{\beta_-} \int \left[\rho^{-\beta_-} L_+^{-1} \left(\frac{1}{\rho^2} \bar{F}(\xi)\right)\right] d\rho + C_- r^{\beta_-} \\
&= r^{\beta_-} \int \left[\rho^{-\beta_-} \rho^{\beta_+ - 1} \int \left(\tau^{-(\beta_+ - 1)} \frac{1}{\tau^2} \bar{F}(\xi) \right)d\tau + \rho^{-\beta_-} C_+ \rho^{\beta_+-1}\right] d\rho + C_- r^{\beta_-} \\
&= r^{\beta_-} \int \left[\rho^{\beta_+ - \beta_- - 1} \int \left(\tau^{-(\beta_+ + 1)} \bar{F}(\xi) \right)d\tau\right] d\rho + \frac{C_+}{\beta_+ - \beta_-} r^{\beta_+} + C_- r^{-\beta_-}.
\end{split}
\]
Letting
\begin{equation}
\label{eq:stationaryC_G}
G(\varphi) = r^{\beta_-} \int \left[\rho^{\beta_+ - \beta_- - 1} \int \left(\tau^{-(\beta_+ + 1)} \bar{F}(\varphi) \right)d\tau\right] d\rho
\end{equation}
and absorbing \(\tfrac{1}{\beta_+ - \beta_-}\) (recall \(\beta_+ \ne \beta_-\)) into \(C_+\), we have
\[
\xi = G(\xi) + C_+ r^{\beta_+} + C_- r^{\beta_-}.
\]
Note that in the case \(\bar{F} = 0\) we have \(G = 0\) and as expected, we recover the general solution, \(\xi = C_+ r{\beta_+} + C_- r^{\beta_-}\) to the linearised equation \(L(\xi) = 0\).

Recalling that \(\beta_- < 0\) and that we require \(\xi(0) = 0\), we set \(C_- = 0\), rewrite \(\lambda = C_+\), and seek a solution of
\begin{equation}
\label{eq:stationaryC_implicit}
\xi = G(\xi) + \lambda r^{\beta_+}
\end{equation}
for arbitrary \(\lambda > 0\).

In order prove the existence and uniqueness of a solution to \eqref{eq:stationaryC_implicit}, we seek a suitable function space to which we may apply a contraction mapping principle, ensuring the existence of a unique fixed point. This is achieved in \Cref{lem:contraction} below, where we prove that for any \(\lambda > 0\) and \(0 < \epsilon < \lambda\), there is an \(R_0 > 0\) such that for any \(r_0 \in (0, R_0)\), there exists a unique \(\xi = \xi_{\epsilon, \lambda, r_0}\) satisfying \eqref{eq:stationaryC_implicit} and such that
\[
\abs{\xi(r) - \lambda r^{\beta_+}} \leq \epsilon r^{\beta_+}, \quad r \in [0, r_0].
\]
Notice that if \(\epsilon' \leq \epsilon\) then \(\xi_{\epsilon', \lambda, r_0}\) satisfies the requirements for \(\xi_{\epsilon, \lambda, r_0}\) hence by uniqueness,
\[
\xi_{\epsilon', \lambda, r_0} = \xi_{\epsilon, \lambda, r_0}, \quad \epsilon' \leq \epsilon
\]
so that \(\xi_{\epsilon, \lambda, r_0}\) is independent of \(\epsilon\). Similar arguments apply for \(r_0' \geq r_0\), so that \(\xi_{\epsilon, \lambda, r_0}\) is independent of both \(\epsilon\) and \(r_0\) providing a unique, self similar solution, \(\xi_{\lambda}\) amongst the class of continuous functions satisfying
\[
\xi = \lambda r^{\beta_+} + \littleo(r^{\beta_+}) \text{ as } r \to 0.
\]
\end{proof}

\begin{rem}
In the case, \(C = 1\), the characteristic equation is \(p(\beta) = \beta^2 - F'(0) = (\beta - \sqrt{F'(0)})(\beta + \sqrt{F'(0)})\), hence \(\beta_+ = -\beta_- = \sqrt{F'(0)}\) and
\[
L_+ = \partial_r - \frac{\beta_+ - 1}{r}, \quad L_- = \partial_r + \frac{\beta_+}{r}.
\]
It seems this difference of squares structure is what allows for the substitution \(r = e^w\) in the \(C=1\) case described above.

The corresponding substitution for \(C \ne 1\) is
\[
r = [1 + (1-C)w]^{\tfrac{1}{1-C}}.
\]
with
\[
\zeta(w) = \xi\left([1 + (1-C)w]^{\tfrac{1}{1-C}}\right).
\]
for \(w \leq w_0 = - \frac{1}{1-C}\), with initial condition \(\zeta(-\infty) = 0\). This change of variables leads to \(\zeta\) satisfying
\begin{equation}
\label{eq:stationaryC_change_var}
\zeta'' = [1 + (1-C)w]^{-2} \left(r^2 \xi'' + C r \xi'\right) = [1 + (1-C)w]^{-2} F(\zeta).
\end{equation}

Note that in the limit,
\[
\lim_{C\to 1} [1 + (1-C)w]^{\tfrac{1}{1-C}} = e^w
\]
and as expected, we recover the change of variables from the \(C = 1\).

Multiplying both sides of \eqref{eq:stationaryC_change_var} by \(\zeta'\) and rearranging, we may rewrite this as
\[
[(\zeta')^2]' = 2 [1 + (1-C)w]^{-2} F(\zeta) \zeta'.
\]
For convenience let \(m(w) = 2[1 + (1-C)w]^{-2}\) so that
\[
[(\zeta')^2]' = m(w) F(\zeta) \zeta'.
\]
Letting \(\eta = \zeta'\) we then have the non-linear system
\[
\begin{cases}
\zeta' &= \eta \\
[\eta^2]' &= m(w) F(\zeta) \eta
\end{cases}
\]

The factor \(m(w)\) now makes direct integration difficult. What's more, \(m(w)\) is not integrable up to \(w_0\) complicating obtaining solutions. Presumably one could formulate a suitable existence result as in \Cref{lem:stationaryC} but there seems no great advantage gained by making the change of variables which is one reason why we work with the original equation. Perhaps more importantly however, is that the proof give for \Cref{lem:stationaryC} shows the influence of the roots \(\beta_{\pm}\) of the characteristic equation \eqref{eq:char_eqn}.
\end{rem}

Now we prove the claimed contraction mapping used in the proof of \Cref{lem:stationaryC}. Recall that \(\bar{F}(z) = \bigo(z^{1+\alpha})\). Then there exists \(z_0 \in (0, a)\) and \(M > 0\) such that,
\begin{equation}
\label{eq:barF_bound}
\abs{\bar{F}(z)} < M z^{1+\alpha}, \quad z \in (0, z_0].
\end{equation}


\begin{lemma}
\label{lem:contraction}
Given \(\lambda > 0\) and \(0 < \epsilon < \lambda\) and
\begin{equation}
\label{eq:r_0}
r_0 < R_0 := \min\left\{\left(\tfrac{z_0}{\lambda + \epsilon}\right)^{1/\beta_+}, \left(\frac{\alpha\beta_+((1+\alpha)\beta_+ - \beta_-)\epsilon}{M (\lambda + \epsilon)^{1+\alpha}}\right)^{1/\alpha\beta_+}\right\}
\end{equation}
define
\begin{equation}
\label{eq:contraction_space}
X = X_{\lambda,\epsilon,r_0} = \left\{\xi \in C([0, 1] \to \R_{\geq 0}) : \forall r \in [0, r_0], \abs{\xi(r) - \lambda r^{\beta_+}} \leq \epsilon r^{\beta_+} \right\}
\end{equation}
and
\begin{equation}
\label{eq:contraction_map}
T(\xi) = T_{\lambda} (\xi) = G(\xi) + \lambda r^{\beta_+}.
\end{equation}

Then,
\begin{enumerate}
\item \(X \subseteq C([0, 1])\) is a closed, convex subset with respect to the sup-norm, \(\|\cdot\|_{\infty}\), \label{itm:closed}
\item \(T : X \to X\), \label{itm:self_map}
\item \(T\) is a strict contraction mapping with respect to the sup-norm, \(\|\cdot\|_{\infty}\). \label{itm:contraction}
\end{enumerate}

Thus by the Banach fixed point theorem, there exists a unique \(\xi \in X\) such that
\[
\xi = G(\xi) + \lambda r^{\beta_+}.
\]
\end{lemma}

\begin{proof}
Point \ref{itm:closed} is straight forward: for each \(z \in \R\) and \(r \in [0, r_0]\), the function
\[
\Phi : (z, r) \mapsto \abs{z - \lambda r^{\beta_+}} - \epsilon r^{\beta_+}
\]
is a continuous map \(\R \times [0, r_0] \to \R\). Now let \((\xi_n)_{n \in \N} \subseteq X\) be such that \(\xi_n \to \xi\) with respect to \(\|\cdot \|_{\infty}\). Then by definition of \(X\), \(\xi_n \geq 0\) and \(\Phi(\xi_n(r), r) \leq 0\). From uniform convergence, for every \(r \in [0, r_0]\) we have \(\xi_n(r) \to \xi(r)\). Thus \(\xi \geq 0\), and by continuity, for every \(r \in [0, r_0]\), \(\Phi(\xi(r), r) \leq 0\) and hence \(\xi \in X\). Convexity follows since if \(\xi_1, \xi_2 \in X\) and \(t \in [0, 1]\), then
\[
\begin{split}
\abs{t\xi_1 + (1-t)\xi_2 - \lambda r^{\beta_+}} &= \abs{t(\xi_1 - \lambda r^{\beta_+}) + (1-t)(\xi_2 - \lambda r^{\beta_+})} \\
&\leq t \abs{\xi_1 - \lambda r^{\beta_+}} + (1-t)\abs{\xi_2 - \lambda r^{\beta_+}} \\
&\leq t \epsilon r^{\beta_+}  + (1-t)\epsilon r^{\beta_+} = \epsilon r^{\beta_+}.
\end{split}
\]

For point \ref{itm:self_map}, from definition \eqref{eq:contraction_map} it suffices to show that for \(\xi \in X\) and all \(r \in [0, r_0]\) we have
\begin{equation}
\label{eq:G_bdd}
\abs{G(\xi)} \leq \epsilon r^{\beta_+}.
\end{equation}
Being more explicit in the bounds of integration in the definition of \(G\) from \eqref{eq:stationaryC_G}, we define
\begin{equation}
\label{eq:G_precise}
G(\xi(r)) = r^{\beta_-} \int_0^r \left[\rho^{\beta_+ - \beta_- - 1} \int_0^{\rho} \left(\tau^{-(\beta_+ + 1)} \bar{F}(\xi(\tau)) \right)d\tau\right] d\rho.
\end{equation}

Now from \(\abs{\xi - \lambda r^{\beta_+}} \leq \epsilon r^{\beta_+}\) in the definition of \(X\) in \eqref{eq:contraction_space}, the choice \(r_0 < \left(\tfrac{z_0}{\lambda + \epsilon}\right)^{1/\beta_+}\) in \eqref{eq:r_0}, and the assumption that \(0 < \epsilon < \lambda\), we have for \(r \in [0, r_0]\) that
\[
0 \leq (\lambda - \epsilon) r^{\beta_+} \leq \xi(r) \leq (\lambda + \epsilon) r^{\beta_+} \leq (\lambda + \epsilon) r_0^{\beta_+} < z_0.
\]
Then by the definition of \(z_0, M\) in \eqref{eq:barF_bound},
\[
\begin{split}
r^{-(\beta_+ + 1)} \abs{\bar{F}(\xi)(r)} &\leq M r^{-(\beta_+ + 1)} \abs{\xi(r)}^{1+\alpha} \leq M r^{-(\beta_+ + 1)} (\lambda + \epsilon)^{1+\alpha} r^{(1+\alpha)\beta_+} \\
&= M(\lambda + \epsilon)^{1+\alpha} r^{\alpha\beta_+ - 1}.
\end{split}
\]
Substitution into \eqref{eq:G_precise} gives,
\[
\begin{split}
\abs{G(\xi(r))} &\leq  M (\lambda + \epsilon)^{1+\alpha} r^{\beta_-} \int_0^r \left[\rho^{\beta_+ - \beta_- - 1} \int_0^{\rho} \tau^{\alpha\beta_+ - 1} d\tau\right] d\rho \\
&= \frac{M (\lambda + \epsilon)^{1+\alpha}}{\alpha\beta_+} r^{\beta_-} \int_0^r \rho^{(1+\alpha)\beta_+ - \beta_- - 1} d\rho \\
&= \frac{M (\lambda + \epsilon)^{1+\alpha}}{\alpha\beta_+((1+\alpha)\beta_+ - \beta_-)} r^{(1+\alpha)\beta_+} \\
&\leq \frac{M (\lambda + \epsilon)^{1+\alpha}r_0^{\alpha\beta_+}}{\alpha\beta_+((1+\alpha)\beta_+ - \beta_-)} r^{\beta_+}.
\end{split}
\]
Keep in mind here that \(\alpha, \beta_+ > 0\) and \(\beta_- < 0\) so all integrals are finite and the constants are positive. Now, equation \eqref{eq:G_bdd} will be true provided
\[
\frac{M (\lambda + \epsilon)^{1+\alpha}r_0^{\alpha\beta_+}}{\alpha\beta_+((1+\alpha)\beta_+ - \beta_-)} \leq \epsilon
\]
which is guaranteed since in equation \eqref{eq:r_0}, we chose \(r_0\) to satisfy
\[
r_0 < \left(\frac{\alpha\beta_+((1+\alpha)\beta_+ - \beta_-)\epsilon}{M (\lambda + \epsilon)^{1+\alpha}}\right)^{1/\alpha\beta_+}.
\]

For point \ref{itm:contraction}, let \(\xi_1, \xi_2 \in X\) and let \(\bar{r} \in (0, r_0]\) be such that
\[
\|G(\xi_1) - G(\xi_2)\|_{\infty} = \abs{G(\xi_1(\bar{r})) - G(\xi_2(\bar{r}))}.
\]
Then
\[
\begin{split}
\|G(\xi_1) - G(\xi_2)\|_{\infty} &\leq \bar{r}^{\beta_-} \int_0^{\bar{r}} \left[\rho^{\beta_+ - \beta_- - 1} \int_0^{\rho} \tau^{-(\beta_+ + 1)} \abs{\bar{F}(\xi_1(\tau)) - \bar{F}(\xi_2(\tau))} d\tau\right] d\rho.
\end{split}
\]
We already know from the asymptotics of \(\bar{F}\) and of \(\xi_1, \xi_2\) that the argument used to prove equation \eqref{eq:G_bdd} shows the inner integral is finite. We estimate
\[
\begin{split}
\int_0^{\rho} \tau^{-(\beta_+ + 1)} \abs{\bar{F}(\xi_1(\tau)) - \bar{F}(\xi_2(\tau))} d\tau &\leq L \int_0^{\rho} \tau^{-(\beta_+ + 1)} \abs{\xi_1(\tau) - \xi_2(\tau)} d\tau \\
&= L \int_0^{\rho} \tau^{-(\beta_+ + 1)} \abs{\xi_1(\tau) - \lambda \tau^{\beta_+} - (\xi_2(\tau) - \lambda \tau^{\beta_+})} d\tau \\
&\leq 2 \epsilon L \int_0^{\rho} \tau^{-1} d\tau \\
\end{split}
\]

{\color{red} Need just a teensy bit more integrability}

To obtain suitable estimates for our purposes here, take any \(s \in (0, \rho)\) and write
\[
\begin{split}
\int_0^{\rho} \tau^{-(\beta_+ + 1)} \abs{\bar{F}(\xi_1(\tau)) - \bar{F}(\xi_2(\tau))} d\tau &= \int_0^{s} \tau^{-(\beta_+ + 1)} \abs{\bar{F}(\xi_1(\tau)) - \bar{F}(\xi_2(\tau))} d\tau \\
&\quad + \int_s^{\rho} \tau^{-(\beta_+ + 1)} \abs{\bar{F}(\xi_1(\tau)) - \bar{F}(\xi_2(\tau))} d\tau \\
&\leq \int_0^{s} \tau^{-(\beta_+ + 1)} \left(\abs{\bar{F}(\xi_1(\tau))} + \abs{\bar{F}(\xi_2(\tau))}\right) d\tau \\
&\quad + L \|\xi_1 - \xi_2\|_{\infty} \int_s^{\rho} \tau^{-(\beta_+ + 1)} d\tau \\
&\leq 2\epsilon s^{\beta_+} + \frac{L}{\beta_+}(s^{-\beta_+} - \rho^{-\beta_+}) \|\xi_1 - \xi_2\|_{\infty}
\end{split}
\]
where \(L\) is the Lipshitz constant of \(\bar{F}(z) = F(z) - F'(0) z\), which is \(C^1\) hence in particular Lipschitz. Here we also used \(\eqref{eq:G_bdd}\) in the last inequality.

Integrating again,
\[
\begin{split}
\|G(\xi_1) - G(\xi_2)\|_{\infty} &\leq \bar{r}^{\beta_-} \int_0^{\bar{r}} \rho^{\beta_+ - \beta_- - 1} \left[2\epsilon s^{\beta_+} + \frac{L}{\beta_+}(s^{-\beta_+} - \rho^{-\beta_+}) \|\xi_1 - \xi_2\|_{\infty} \right] d\rho \\
&= \frac{1}{\beta_+ - \beta_-} \left(2\epsilon s^{\beta_+} + \frac{L}{\beta_+} s^{-\beta_+} \|\xi_1 - \xi_2\|_{\infty}\right) \bar{r}^{\beta_+}  + \frac{1}{\beta_-} \|\xi_1 - \xi_2\|_{\infty} \\
&\leq \frac{1}{\beta_+ - \beta_-} \left(2\epsilon s^{\beta_+} + \frac{L}{\beta_+} s^{-\beta_+} \|\xi_1 - \xi_2\|_{\infty}\right) \bar{r}^{\beta_+} \\
&= \frac{1}{\beta_+ - \beta_-} \left(2\epsilon \bar{r}^{2\beta_+} + \frac{L}{\beta_+} \|\xi_1 - \xi_2\|_{\infty}\right) \left(\frac{\bar{r}}{s}\right)^{\beta_+} \\
\end{split}
\]
since \(\beta_- \leq 0\).


\hrule

For point \ref{itm:contraction}, we also make use of equation \eqref{eq:G_bdd}. Let \(\xi_1, \xi_2 \in X\). We wish to show that there exists an \(L \in [0, 1)\) such that
\[
\|G(\xi_1) - G(\xi_2)\|_{\infty} \leq L\|\xi_1 - \xi_2\|_{\infty}.
\]
Equation \eqref{eq:G_bdd} and the strict inequality in the definition of \(r_0\) in \eqref{eq:r_0} gives
\[
|G(\xi_1(r)) - G(\xi_2(r))| \leq |G(\xi_1(r))| + |G(\xi_2(r))| < 2\epsilon r^{\beta_+} \leq 2 \epsilon r_0^{\beta_+}
\]
so that
\[
\|G(\xi_1) - G(\xi_2)\|_{\infty} < 2 \epsilon r_0^{\beta_+}.
\]
On the other hand, the definition of \(X\) in equation \eqref{eq:contraction_space} ensures that for \(i = 1,2\),
\[
(\lambda - \epsilon) r^{\beta_+} \leq \xi_i(r) \leq (\lambda + \epsilon) r^{\beta_+}.
\]
Therefore,
\[
-2\epsilon r^{\beta_+} = (\lambda - \epsilon) r^{\beta_+} - (\lambda + \epsilon) r^{\beta_+} \leq \xi_1(r) - \xi_2(r) \leq (\lambda + \epsilon) r^{\beta_+} - (\lambda - \epsilon) r^{\beta_+} = 2 \epsilon r^{\beta_+}.
\]
That is
\[
\abs{\xi_1(r) - \xi_2(r)} \leq 2\epsilon r^{\beta_+}.
\]
{\color{red} Not much use.}
{\color{red} This is the closest I can get just now:}

We do have
\[
\xi(r) \geq (\lambda - \epsilon) r^{\beta_+} < \frac{\lambda}{2} r^{\beta_+}
\]
if we choose \(\epsilon < \tfrac{\lambda}{2}\). Then
\[
\|\xi\|_{\infty} \geq \frac{\lambda}{2} r_0^{\beta_+} >  \epsilon r_0^{\beta_+} \geq \|G(\xi)(r)\|_{\infty}
\]
by equation \eqref{eq:G_bdd}. Therefore, \(G\) is strictly norm decreasing on \(X\).
\end{proof}

\section{Blow up}

Recall, our differential equation is
\[
\partial_t \psi = \psi_{rr} + \frac{C}{r} \psi_r - \frac{1}{r^2} F(\psi)
\]
Here we show that for bounded solutions, the phenomena of blow depends only on \(C\) and \(\lim_{z\to 0} z^{-1} F(z)\). Since we assume \(F\) is Lipschitz, we have \(|z^{-1} F(z)|\) is bounded near \(0\) and we now make the further assumption that the limit exists in which case, since \(F(0) = 0\) and \(F\) is odd and increasing,
\[
A = \lim_{z\to 0} z^{-1} F(z) \geq 0.
\]
The model case is if \(F\) is differentiable at \(z=0\), in which case the limit is simply \(A = F'(0)\) and if \(F\) is say \(C^2\) near \(z=0\), then \(F(z) = Az + \mathcal{O} (z^2)\) as \(z \to 0\).

\subsection{Comparison Principle}

Our argument is based on the following comparison principle:

\begin{thm}[Comparison Principle]
\label{thm:comparison}
Let \(\psi\) satisfy equation \eqref{eq:pde} and let \(\overline{\xi}\) be a super-solution; that is
\begin{equation}
\label{eq:pde_super}
\begin{cases}
\partial_t \overline{\xi} &\geq \overline{\xi}_{rr} + \frac{C}{r} \overline{\xi}_r - \frac{1}{r^2} F(\overline{\xi}) \\
\overline{\xi}(r, 0) &\geq \psi_0(r) \\
\overline{\xi}(0, t) &= 0, \quad \overline{\xi}(1, t) \geq \psi_1, \quad t \in [0, T).
\end{cases}
\end{equation}
Then \(\psi \leq \overline{\xi}\) for all \(t \in [0, T)\). If on the other hand, \(\underline{\xi}\) satisfies the conditions of equation \eqref{eq:pde_super}, but with all inequalities reversed, then \(\psi \geq \underline{\xi}\).
\end{thm}

The proof is basically a standard comparison principle, but with a slight adjustment to deal with the singular behaviour of the coefficients \(1/r\) and \(F(\psi(r))/r^2\) near \(r=0\). Note also that the only non-linearity occurs in the lowest order term \(F(\psi(r))/r^2\). These constraints account for the assumptions on \(F\) given in the introduction (see equations \eqref{eq:near_positive} and \eqref{eq:far_positive}).

\begin{proof}
We only prove the super-solution case, \(\overline{\xi}\). The sub-solution case \(\underline{\xi}\) is similar.

Let
\[
\varphi = \psi - \overline{\xi}.
\]
Then the initial assumption is \(\varphi(r, 0) \leq 0\) and the claim of the theorem is that \(\varphi(r, t) \leq 0\) for all \(t \in [0, T)\). As usual we argue by contradiction: If the claim is false, then there is a \(\tau > 0\) such that
\begin{equation}
\label{eq:false_claim}
\sup \{\varphi(r, t) : 0 \leq r \leq 1, 0 \leq t \leq \tau\} > 0.
\end{equation}

Before going through the contradiction argument, we first need to handle the singular coefficients at \(r = 0\). Since \(\psi(0, t) = \overline{\xi}(0, t) = 0\), by the regularity of \(\psi\) and \(\overline{\xi}\), there exists a \(\rho > 0\) such that
\[
-\delta < \psi, \xi < \delta, \quad \text{for} \quad 0 \leq r \leq \rho, \quad 0 \leq t \leq \tau.
\]
By the assumptions \(F(0) = 0\), and \(F\) is odd, increasing on \((-\delta,\delta)\), we must have
\begin{equation}
\label{eq:near_positive}
0 \leq \frac{F(\psi) - F(\overline{\xi})}{\psi - \overline{\xi}}, \quad (r, t) \in [0, \rho] \times [0, \tau].
\end{equation}

Next, to handle \(r\) away from \(\rho\), we choose any \(\lambda\) such that
\[
\lambda > \frac{1}{\rho^2} \operatorname{Lip} (F) > 0
\]
where \(\operatorname{Lip} (F)\) is the Lipschitz constant of \(F\). Then for  \(r \geq \rho\) we have
\begin{equation}
\label{eq:far_positive}
\lambda + \frac{1}{r^2} \frac{F(\psi) - F(\overline{\xi})}{\psi - \overline{\xi}} > \lambda - \frac{1}{r^2} \operatorname{Lip} (F) > 0
\end{equation}

Now we may define
\[
h = e^{-\lambda t} \varphi.
\]
Note that \(\tau\) is already given by assuming the theorem is false in \eqref{eq:false_claim} and is independent of \(\lambda\) which is essential. A lower bound for the constant \(\lambda\) does depend on \(\tau\) through \(\rho\) but we are free to increase \(\lambda\).

We may now apply the usual contradiction argument as follows: The assumptions of the theorem give
\[
h(0, t) \leq 0, \quad h(1, t) \leq 0 \quad \text{and} \quad h(r, 0) \leq 0
\]
while claiming the conclusion of the theorem is false implies, by equation \eqref{eq:false_claim} that
\[
\sup \{h(r, t) : 0 \leq r \leq 1, 0 \leq t \leq \tau\} > 0.
\]
Let \((r_0, t_0) \in (0, 1) \times (0, \tau]\) realise the supremum. Then at \((r_0, t_0)\) we have
\begin{equation}
\label{eq:max_principle_inequality}
\begin{split}
\partial_t h &= - \lambda e^{-\lambda t} \varphi + e^{-\lambda t} \partial_t \varphi \\
&\leq  -\lambda e^{-\lambda t} \varphi + e^{-\lambda t} \left[\psi_{rr} + \frac{C}{r} \psi_r - \frac{1}{r^2} F(\psi)\right] \\
&\quad - e^{-\lambda t}\left[\overline{\xi}_{rr} + \frac{C}{r} \overline{\xi}_r - \frac{1}{r^2} F(\overline{\xi})\right] \\
&= h_{rr} + \frac{C}{r} h_r - \left(\lambda + \frac{1}{r^2} \frac{F(\psi) - F(\overline{\xi})}{\psi - \overline{\xi}}\right) h.
\end{split}
\end{equation}
Moreover, \(h(r_0, t_0)\) equals the supremum which is assumed positive, hence by equation \eqref{eq:max_principle_inequality},
\[
\lambda + \frac{1}{r^2} \frac{F(\psi) - F(\overline{\xi})}{\psi - \overline{\xi}} \leq \frac{1}{h} \left(h_{rr} + \frac{C}{r} h_r - \partial_t h\right) \leq 0.
\]
But this gives a contradiction since using equation \eqref{eq:near_positive} for \(r_0 \leq \rho\), and equation \eqref{eq:far_positive} for \(r_0 \geq \rho\) gives
\[
\lambda + \frac{1}{r^2} \frac{F(\psi) - F(\overline{\xi})}{\psi - \overline{\xi}} > 0.
\]
\end{proof}

\subsection{Blow up at the origin}
\label{subsec:origin_blowup}

With our heuristic motivation to hand, we move on to show that blow up of bounded solutions, if it occurs at all, must occur at the origin. This is a straight forward application of standard estimates \cite[Theorem 10.1]{Ladyzhenskaja:/1967} but we record the details in the following lemma for convenience.

{\color{red} Might need to clarify norms here. They probably should be spatial norms and we take a sup over t. Also $L^{\infty}$ bounds might follow from parabolic theory and bounds on the initial data.}

\begin{lemma}
\label{lem:apriori_bounds}
Let \(\psi\) be a solution of \eqref{eq:pde} with initial data \(\psi_0\) satisfying
\[
\|\psi_0\|_{C^{2,\alpha}([0, 1])} < \infty,
\]
and
\[
\|\psi\|_{L^{\infty} ([0, 1] \times [0, \tau])} < \infty
\]
If there exists a \(\rho \in (0, 1]\) and a \(\tau \in (0, T)\) such that
\[
\|\psi\|_{C^{2,\alpha}([0, \rho] \times [0, \tau])} < \infty,
\]
then
\[
\|\psi\|_{C^{2,\alpha}([0, 1] \times [0, \tau])} < \infty.
\]
In particular, if the maximal existence time \(T < \infty\), then
\[
\lim_{\tau\to T} \inf_{\rho \in (0, 1)} \{\|\psi\|_{C^2([0, \rho] \times [0, \tau])}\} = \infty.
\]
\end{lemma}

\begin{proof}
We need to show
\[
\|\psi\|_{C^{2,\alpha}([\rho, 1] \times [0, \tau])} < \infty,
\]
under the hypothesis
\[
\|\psi\|_{C^{2,\alpha}([0, \rho] \times [0, \tau])} < \infty.
\]

We use the notation of \cite[Theorem 10.1]{Ladyzhenskaja:/1967}. Define the linear operator
\[
\mathcal{L} \psi = \partial_t \psi - \psi_{rr} - \frac{C}{r} \psi_r
\]
and
\[
f(r, t) = -\frac{1}{r^2} F(\psi(r, t))
\]
so that
\[
\mathcal{L} \psi = f.
\]

Let \(\Omega = (0, 1)\), \(\Omega' = (\rho, 1)\), and \(\Omega'' = (\rho/2, 1)\), \(S = \{0, 1\}\), \(S' = S'' = \{1\}\). \cite[Theorem 10.1]{Ladyzhenskaja:/1967} gives constants \(c_1, c_2 > 0\) such that
\[
\begin{split}
\|\psi\|_{C^{2,\alpha}([\rho, 1] \times [0, \tau])} &\leq c_1 \left(\|\tfrac{1}{r^2} F(\psi)\|_{C^{0,\alpha}([\rho/2, 1] \times [0, \tau])} + \|\psi_0\|_{C^{2,\alpha}([\rho/2, 1])} + \|\psi\|_{C^{2,\alpha}(\{1\} \times [0, \tau])} \right) \\
&\quad + c_2 \|\psi\|_{C([\rho/2, 1] \times [0, \tau])}.
\end{split}
\]

In the hypothesis of the theorem, we assume \(\psi\) has finite \(C^{0,\alpha}\) norm, hence the last term is finite. Since \(F\) is also Lipschitz, the first term \(r^{-2} F(\psi(r, t))\) has finite \(C^{0,\alpha}\) norm on \(r \in [\rho, 1], t \in [0, \tau]\). The second term is the initial data \(\psi_0\) which is assumed to have finite \(C^{2,\alpha}\) norm and finally the third term - the boundary term \(\psi|_{r=1}\) - is constant hence also has finite norm.
\end{proof}

\subsection{Criteria for blow up}

The following lemma, simply combines what we have obtained so far into a criteria for blow up depending on the existence of suitable barriers.

\begin{lemma}[Blow up criteria]
\label{lem:blowup_criteria}
Let \(\psi\) solve the equivariant harmonic map heat flow, equation \eqref{eq:pde} on \([0, 1] \times [0, T)\). Then  we have the following:

\begin{enumerate}
\item If there exists a sub-solution \(\xi\) to equation \eqref{eq:pde} such that \(\xi(r, t=0) \leq \psi(r, t=0)\) and \(\xi'(0, t) \to \infty\) as \(t \to T\), then \(\psi'(0, t) \to \infty\) as \(t \to T\). Likewise, if \(\psi(r, t = 0) \leq \xi(r, t=0)\) for a super-solution with \(\xi'(0, t) \to -\infty\) as \(t \to T\), then \(\psi'(0, t) \to \infty\) as \(t \to T\). \label{itm:blowup}
\item If there exists a super-solution \(\xi\) to equation \eqref{eq:pde} such that \(-\xi(r, t=0) \leq \psi(r, t=0) \leq \xi(r, t-0)\) with \(\abs{\xi} \leq M_1\) and \(\abs{\xi'(0, t)} \leq M_2 < \infty\) for \(t \in [0, T)\) and some \(M_1, M_2 > 0\) then there exists an \(N\) with \(0 < N < \infty\) such that \(\|\psi(\cdot, t)\|_{C^{1,\alpha}} \leq N\) for all \(t \in [0, T)\). \label{itm:noblowup}
\end{enumerate}
\end{lemma}

\begin{proof}
Proof of \ref{itm:blowup}:

We just prove the sub-solution and blow up to \(\infty\) case, the super-solution and blow up to \(-\infty\) being identical and in fact following by replacing \(\xi\) with \(-\xi\).

By the comparison principle \Cref{thm:comparison}, \(\psi(r, t) \geq \xi(r, t)\) for all \(r \in [0, 1]\) and \(t \in [0, T)\). Since \(\psi(0, t) = \xi(0, t)\) we then must have \(\psi'(0, t) \geq \xi'(0, t) \to \infty\) as \(t \to T\).

Proof of \ref{itm:noblowup}:

Just as with the previous case the comparison principle \Cref{thm:comparison} gives \(-\xi(r, t) \leq \psi(r, t) \leq \xi(r, t)\) for all \(r \in [0, 1]\) and \(t \in [0, T)\) and hence \(\abs{\psi(r, t)} \leq M_1\) and \(-M_2 \leq -\xi'(0, t) \leq \psi'(0, t) \leq \xi'(0, t) \leq M_1\). Thus for any \(\tau \in [0, T)\) we have a \(\rho_0 > 0\) with
\[
\|\psi(\cdot, t)\|_{C^{1,\alpha}([0, \rho_0])} \leq 2M_2.
\]
for all \(t \in [0, \tau]\). Then \Cref{lem:apriori_bounds} ensures that
\[
\|\psi(\cdot, t)\|_{C^{1,\alpha}([0, 1])} < N
\]
for all \(t \in [0, \tau]\) with \(N\) depending only on the local \(C^1\) bound \(M_2\), the initial data and the \(C^0\) bound \(M_1\).
\end{proof}

\begin{rem}
In the case \ref{itm:noblowup}, if \(T < \infty\), then we have uniform \(C^{1,\alpha}\) bounds on \([0, T)\) and thus the solution extends past \(T\). In particular, if \(\xi\) is defined on all \([0, \infty)\) with the same assumptions as in the lemma, then no finite time blow up can occur.
\end{rem}

\section{Barriers}
\label{subsec:barriers}

Now we may construct barriers from the stationary solutions given in \Cref{lem:stationaryC1,lem:stationaryC} to which we apply the blow up criteria \Cref{lem:blowup_criteria} to characterise whether blow up occurs or not.

\subsection{The case \(\|\psi\|_{\infty} < a, C=1, \sqrt{F'(0)} > 1, \psi_0 = \bigo(r^{\sqrt{F'(0)}})\)}

This is the simplest case, and we can just use the stationary solution from \Cref{lem:stationaryC1} as a barrier. Recall that \(\xi_{\lambda} (r) = \sqrt{F'(0)} \lambda r^{\beta_+} + \bigo(r^{\beta + \epsilon})\) and \(\lim_{\lambda \to \infty} \xi_{\lambda} (r) = \infty\). Thus we may choose \(\lambda\) large enough such that \(\xi_{\lambda}(r) \geq \psi_0(r)\). Then since \(\xi\) is a stationary solution, it is in particular a super solution hence \Cref{lem:blowup_criteria} applies to rule out finite time blow up. 

\end{document}
