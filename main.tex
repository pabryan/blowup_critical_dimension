\documentclass{amsart}

%\usepackage{etoolbox}
%\makeatletter
%\let\ams@starttoc\@starttoc
%\makeatother
%\makeatletter
%\let\@starttoc\ams@starttoc
%\patchcmd{\@starttoc}{\makeatletter}{\makeatletter\parskip\z@}{}{}
%\makeatother

%\usepackage[parfill]{parskip}

\usepackage[colorlinks=true,linkcolor=blue,citecolor=blue,urlcolor=blue]{hyperref}
\usepackage{bookmark}
\usepackage{amsthm,thmtools,amssymb,amsmath,amscd}

\usepackage[bibstyle=alphabetic,citestyle=alphabetic,backend=bibtex]{biblatex}
\bibliography{Bibliography}

\usepackage{fancyhdr}
\usepackage{esint}

\usepackage{enumerate}

\usepackage{pictexwd,dcpic}

\usepackage{graphicx}

\swapnumbers
\declaretheorem[name=Theorem,numberwithin=section]{thm}
\declaretheorem[name=Remark,style=remark,sibling=thm]{rem}
\declaretheorem[name=Lemma,sibling=thm]{lemma}
\declaretheorem[name=Proposition,sibling=thm]{prop}
\declaretheorem[name=Definition,style=definition,sibling=thm]{defn}
\declaretheorem[name=Corollary,sibling=thm]{cor}
\declaretheorem[name=Assumption,style=remark,sibling=thm]{ass}
\declaretheorem[name=Example,style=remark,sibling=thm]{example}


\numberwithin{equation}{section}

\usepackage{cleveref}
\crefname{lemma}{Lemma}{Lemmata}
\crefname{prop}{Proposition}{Propositions}
\crefname{thm}{Theorem}{Theorems}
\crefname{cor}{Corollary}{Corollaries}
\crefname{defn}{Definition}{Definitions}
\crefname{example}{Example}{Examples}
\crefname{rem}{Remark}{Remarks}
\crefname{ass}{Assumption}{Assumptions}
\crefname{not}{Notation}{Notation}

%Symbols
\renewcommand{\~}{\tilde}
\renewcommand{\-}{\bar}
\newcommand{\bs}{\backslash}
\newcommand{\cn}{\colon}
\newcommand{\sub}{\subset}

\newcommand{\N}{\mathbb{N}}
\newcommand{\R}{\mathbb{R}}
\newcommand{\Z}{\mathbb{Z}}
\renewcommand{\S}{\mathbb{S}}
\renewcommand{\H}{\mathbb{H}}
\newcommand{\C}{\mathbb{C}}
\newcommand{\K}{\mathbb{K}}
\newcommand{\Di}{\mathbb{D}}
\newcommand{\B}{\mathbb{B}}
\newcommand{\8}{\infty}

%Greek letters
\renewcommand{\a}{\alpha}
\renewcommand{\b}{\beta}
\newcommand{\g}{\gamma}
\renewcommand{\d}{\delta}
\newcommand{\e}{\epsilon}
\renewcommand{\k}{\kappa}
\renewcommand{\l}{\lambda}
\renewcommand{\o}{\omega}
\renewcommand{\t}{\theta}
\newcommand{\s}{\sigma}
\newcommand{\p}{\varphi}
\newcommand{\z}{\zeta}
\newcommand{\vt}{\vartheta}
\renewcommand{\O}{\Omega}
\newcommand{\D}{\Delta}
\newcommand{\G}{\Gamma}
\newcommand{\T}{\Theta}
\renewcommand{\L}{\Lambda}

%Mathcal Letters
\newcommand{\cL}{\mathcal{L}}
\newcommand{\cT}{\mathcal{T}}
\newcommand{\cA}{\mathcal{A}}
\newcommand{\cW}{\mathcal{W}}

%Mathematical operators
\newcommand{\INT}{\int_{\O}}
\newcommand{\DINT}{\int_{\d\O}}
\newcommand{\Int}{\int_{-\infty}^{\infty}}
\newcommand{\del}{\partial}

\newcommand{\inpr}[2]{\left\langle #1,#2 \right\rangle}
\newcommand{\fr}[2]{\frac{#1}{#2}}
\newcommand{\x}{\times}
\DeclareMathOperator{\Tr}{Tr}

\DeclareMathOperator{\dive}{div}
\DeclareMathOperator{\id}{id}
\DeclareMathOperator{\pr}{pr}
\DeclareMathOperator{\Diff}{Diff}
\DeclareMathOperator{\supp}{supp}
\DeclareMathOperator{\graph}{graph}
\DeclareMathOperator{\osc}{osc}
\DeclareMathOperator{\const}{const}
\DeclareMathOperator{\dist}{dist}
\DeclareMathOperator{\loc}{loc}
\DeclareMathOperator{\grad}{grad}
\DeclareMathOperator{\ric}{Ric}
\DeclareMathOperator{\Rm}{Rm}
\DeclareMathOperator{\weingarten}{\mathcal{W}}
\DeclareMathOperator{\inj}{inj}

%Environments
\newcommand{\Theo}[3]{\begin{#1}\label{#2} #3 \end{#1}}
\newcommand{\pf}[1]{\begin{proof} #1 \end{proof}}
\newcommand{\eq}[1]{\begin{equation}\begin{alignedat}{2} #1 \end{alignedat}\end{equation}}
\newcommand{\IntEq}[4]{#1&#2#3	 &\quad &\text{in}~#4,}
\newcommand{\BEq}[4]{#1&#2#3	 &\quad &\text{on}~#4}
\newcommand{\br}[1]{\left(#1\right)}

\newcommand{\abs}[1]{\left|{#1}\right|}


%Logical symbols
\newcommand{\Ra}{\Rightarrow}
\newcommand{\ra}{\rightarrow}
\newcommand{\hra}{\hookrightarrow}
\newcommand{\mt}{\mapsto}

%Notes
\newcommand{\pa}[1]{{\color{green} pa: {#1}}}
\newcommand{\jj}[1]{{\color{red} jj: {#1}}}
\newcommand{\mni}[1]{{\color{blue} mni: {#1}}}

%Fonts
\newcommand{\mc}{\mathcal}
\renewcommand{\it}{\textit}
\newcommand{\mrm}{\mathrm}

%Spacing
\newcommand{\hp}{\hphantom}


%\parindent 0 pt

\protected\def\ignorethis#1\endignorethis{}
\let\endignorethis\relax
\def\TOCstop{\addtocontents{toc}{\ignorethis}}
\def\TOCstart{\addtocontents{toc}{\endignorethis}}


\begin{document}

\title[]
 {Blow up of equivariant harmonic map heat flow and Yang Mills flow}

\curraddr{}
\email{}

\dedicatory{}
\subjclass[2010]{}
\keywords{}

\begin{abstract}
\end{abstract}

\maketitle

\section{Introduction}
\label{sec:intro}

Consider a warped product metric of the form,
\[
ds^2 = dz^2 + u(z)^2 d\phi^2
\]
on $M = \{(z,\phi) \in (0, a) \times (0, 2\pi)\}$. The metric extends smoothly to an open disc of radius $a > 0$ in $\R^2$ provided
\[
\begin{cases}
u(0) &= 0 \\
u'(0) &= 1 \\
u(-z) &= - u(z).
\end{cases}
\]
If in addition,
\[
\begin{cases}
u(a) &= 0 \\
u'(a) &= -1 \\
\end{cases}
\]
the the metric extends smoothly to $\S^2$. If $u'(a) \ne 0$, then we still obtain a metric on $\S^2$ but with a conical singularity at $a$.

We consider maps
\[
f : B \times [0, T) \to (\S^2, ds^2)
\]
where $B \subset \R^2$ is the unit ball equipped with the flat metric. Let us write $(r, \theta)$ for polar coordinates on $\R^2$ identifying $B \backslash \{0\}$ with $((0, 1) \times (0, 2\pi), dr^2 + r^2 d\theta^2)$. For a non-zero natural number \(\ell \in \N\), we say $f$ is \emph{degree $\ell$-equivariant} if
\[
f (r, \theta, t) = (\psi(r, t), \ell\theta (\operatorname{mod} 2\pi)).
\]
for some smooth $\psi : (0, 1) \times [0, T) \to (0, a)$ with \(\psi(0, t) = 0\) for every \(t \in [0, T)\).

We are interested in the phenomena of \emph{blow-up} of the \emph{harmonic map heat flow}, for which purpose it is sufficient to fix the boundary data, \(\psi(1, t) = \psi_1\) for a real constant \(\psi_1 > 0\). Of course the initial condition for \(f_0(r, \theta) = f(r, \theta, 0)\) is also equivariant so we take initial condition, \(\psi_0\) with \(\psi_0(0) = 0\) and \(\psi_0(1) = \psi_1\).

The harmonic map heat flow with the stated boundary conditions then becomes
\begin{equation}
\label{eq:harmonic_map_heat_flow}
\begin{cases}
\partial_t \psi &= \psi_{rr} + \frac{1}{r} \psi_r - \frac{\ell^2}{r^2} F(\psi) \\
\psi(r, 0) &= \psi_0(r) \\
\psi(0, t) &= 0, \quad \psi(1, t) = \psi_1, \quad t \in [0, T)
\end{cases}
\end{equation}
where \(F(z) = u(z)u'(z)\).

\subsection*{Examples}

\subsubsection*{Harmonic maps into the sphere.}

The round metric on the sphere is
\[
ds^2 = dz^2 + \sin^2(z) d\phi^2
\]
for \(z \in (0, \pi)\) and where \(d\phi^2\) is the round metric on \(\S^{n-1}\). The corresponding harmonic map heat flow is
\[
\partial_t \psi = \psi_{rr} + \frac{n-1}{r} \psi_r - \frac{n-1}{r^2} \frac{1}{2} \sin (2\psi).
\]
The so-called \emph{critical dimension} here is \(n=2\) for which the harmonic map heat flow is conformally invariant. In this case,
\[
\partial_t \psi = \psi_{rr} + \frac{1}{r} \psi_r - \frac{1}{r^2} \frac{1}{2} \sin (2\psi)
\]
and \(F(z) = \tfrac{1}{2} \sin(2z)\) in equation \eqref{eq:harmonic_map_heat_flow}.

\subsubsection*{\(SO(4)\)-equivariant Yang-Mills flow in \(\R^4\).}

Equivariant Yang-Mills flow reduces to
\[
\partial_t \psi = \psi_{rr} + \frac{n-3}{r} \psi_r - \frac{2(n-2)}{r^2} \psi(1 - \psi)(1 - \tfrac{1}{2} \psi).
\]

Here the critical dimension is \(n=4\) for which the Yang-Mills flow is conformally invariant. In this case,
\[
\partial_t \psi = \psi_{rr} + \frac{1}{r} \psi_r - \frac{1}{r^2} 4 \psi(1 - \psi)(1 - \tfrac{1}{2} \psi)
\]
and \(F(z) = 4 z(1 - z)(1 - \tfrac{1}{2} z)\) in equation \eqref{eq:harmonic_map_heat_flow}.

\section{Comparison Principle}

Our argument is based on a comparison principle. In this section, let us consider the more general equation
\begin{equation}
\label{eq:pde}
\begin{cases}
\partial_t \psi &= \psi_{rr} + \frac{C_1}{r} \psi_r - \frac{C_2}{r^2} F(\psi) \\
\psi(r, 0) &= \psi_0(r) \\
\psi(0, t) &= 0, \quad \psi(1, t) = \psi_1, \quad t \in [0, T)
\end{cases}
\end{equation}
where \(C_1, C_2 > 0\) are real constants and \(F\) is a Lipschitz function that is odd and increasing on some interval \((-\delta, \delta)\) around \(0\). Note that \(F(0) = 0\) since \(F\) is odd. A particular example is any \(C^2\), odd function \(F\) such that \(F(0)\) and \(F'(0) > 0\).

\begin{thm}[Comparison Principle]
Let \(\psi\) satisfy equation \eqref{eq:pde} and let \(\overline{\xi}\) be a super-solution; that is
\begin{equation}
\label{eq:pde_super}
\begin{cases}
\partial_t \overline{\xi} &\geq \overline{\xi}_{rr} + \frac{C_1}{r} \overline{\xi}_r - \frac{C_2}{r^2} F(\overline{\xi}) \\
\overline{\xi}(r, 0) &\geq \psi_0(r) \\
\overline{\xi}(0, t) &\geq 0, \quad \overline{\xi}(1, t) \geq \psi_1, \quad t \in [0, T).
\end{cases}
\end{equation}
Then \(\psi \leq \overline{\xi}\) for all \(t \in [0, T)\). If on the other hand, \(\underline{\xi}\) satisfies the conditions of equation \eqref{eq:pde_super}, but with all inequalities reversed, then \(\psi \geq \underline{\xi}\).
\end{thm}

The proof is basically a standard comparison principle, but with a slight adjustment to deal with the singular behaviour of the coefficients \(1/r\) and \(F(\psi(r))/r^2\) near \(r=0\). Note also that the only non-linearity occurs in the lowest order term \(F(\psi(r))/r^2\). These constraints account for the assumptions on \(F\) (see equations \eqref{eq:near_positive} and \eqref{eq:far_positive}).

\begin{proof}
We only prove the super-solution case, \(\overline{\xi}\). The sub-solution case \(\underline{\xi}\) is similar.

Let \(\varphi = \psi - \overline{\xi}\). Then the initial assumption is \(\varphi(r, 0) \leq 0\) and the claim of the theorem is that \(\varphi(r, t) \leq 0\) for all \(t \in [0, T)\). As usual we argue by contradiction: If the claim is false, then there is a \(\tau > 0\) such that
\begin{equation}
\label{eq:false_claim}
\sup \{\varphi(r, t) : 0 \leq r \leq 1, 0 \leq t \leq \tau\} > 0.
\end{equation}

Since \(\psi(0, t) = \overline{\xi}(0, t) = 0\), by the regularity of \(\psi\) and \(\overline{\xi}\), there exists a \(\rho > 0\) such that
\[
-\delta < \psi, \xi < \delta
\]
for \(0 \leq r \leq \rho\) and \(0 \leq t \leq \tau\). By the assumptions \(F(0) = 0\), and \(F\) is odd, increasing on \((-\delta,\delta)\), we must have
\begin{equation}
\label{eq:near_positive}
0 \leq \frac{F(\psi) - F(\overline{\xi})}{\psi - \overline{\xi}}, \quad (r, t) \in [0, \rho] \times [0, \tau].
\end{equation}
Choose any
\[
\lambda > \frac{C_2}{\rho^2} \operatorname{Lip} (F) > 0
\]
where \(\operatorname{Lip} (F)\) is the Lipschitz constant of \(F\). Then for  \(r \geq \rho\) we have
\begin{equation}
\label{eq:far_positive}
\lambda + \frac{C_2}{r^2} \frac{F(\psi) - F(\overline{\xi})}{\psi - \overline{\xi}} > \lambda - \frac{C_2}{r^2} \operatorname{Lip} (F) > 0
\end{equation}

Now we may define
\[
h = e^{-\lambda t} \varphi.
\]
and apply the usual contradiction argument as follows: The assumptions of the theorem give
\[
h(0, t) \leq 0, \quad h(1, t) \leq 0 \quad \text{and} \quad h(r, 0) \leq 0
\]
while claiming the conclusion of the theorem is false implies, by equation \eqref{eq:false_claim}, that
\[
\sup \{h(r, t) : 0 \leq r \leq 1, 0 \leq t \leq \tau\} > 0.
\]
Let \((r_0, t_0) \in (0, 1) \times (0, \tau]\) realise the supremum. Computing, we see that \(h\) satisfies
\[
\begin{split}
\partial_t h &= - \lambda e^{-\lambda t} \varphi + e^{-\lambda t} \partial_t \varphi \\
&\leq  -\lambda e^{-\lambda t} \varphi + e^{-\lambda t} \left[\psi_{rr} + \frac{C_1}{r} \psi_r - \frac{C_2}{r^2} F(\psi)\right] \\
&\quad - e^{-\lambda t}\left[\overline{\xi}_{rr} + \frac{C_1}{r} \overline{\xi}_r - \frac{C_2}{r^2} F(\overline{\xi})\right] \\
&= h_{rr} + \frac{C_1}{r} h_r - \left(\lambda + \frac{C_2}{r^2} \frac{F(\psi) - F(\overline{\xi})}{\psi - \overline{\xi}}\right) h.
\end{split}
\]
At \((r_0, t_0)\) realising the supremum, \(h(r_0, t_0) > 0\), and hence
\[
\lambda + \frac{C_2}{r^2} \frac{F(\psi) - F(\overline{\xi})}{\psi - \overline{\xi}} \leq \frac{1}{h} \left(h_{rr} + \frac{C_1}{r} h_r - \partial_t h\right) \leq 0.
\]
But this gives a contradiction since using equation \eqref{eq:near_positive} for \(r_0 \leq \rho\), and equation \eqref{eq:far_positive} for \(r_0 \geq \rho\) gives
\[
\lambda + \frac{C_2}{r^2} \frac{F(\psi) - F(\overline{\xi})}{\psi - \overline{\xi}} > 0.
\]
\end{proof}

\end{document}
