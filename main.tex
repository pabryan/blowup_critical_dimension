\documentclass{amsart}

%\usepackage{etoolbox}
%\makeatletter
%\let\ams@starttoc\@starttoc
%\makeatother
%\makeatletter
%\let\@starttoc\ams@starttoc
%\patchcmd{\@starttoc}{\makeatletter}{\makeatletter\parskip\z@}{}{}
%\makeatother

%\usepackage[parfill]{parskip}

\usepackage[colorlinks=true,linkcolor=blue,citecolor=blue,urlcolor=blue]{hyperref}
\usepackage{bookmark}
\usepackage{amsthm,thmtools,amssymb,amsmath,amscd}

\usepackage[bibstyle=alphabetic,citestyle=alphabetic,backend=bibtex]{biblatex}
\bibliography{Bibliography}

\usepackage{fancyhdr}
\usepackage{esint}

\usepackage{enumerate}

\usepackage{pictexwd,dcpic}

\usepackage{graphicx}

\swapnumbers
\declaretheorem[name=Theorem,numberwithin=section]{thm}
\declaretheorem[name=Remark,style=remark,sibling=thm]{rem}
\declaretheorem[name=Lemma,sibling=thm]{lemma}
\declaretheorem[name=Proposition,sibling=thm]{prop}
\declaretheorem[name=Definition,style=definition,sibling=thm]{defn}
\declaretheorem[name=Corollary,sibling=thm]{cor}
\declaretheorem[name=Assumption,style=remark,sibling=thm]{ass}
\declaretheorem[name=Example,style=remark,sibling=thm]{example}


\numberwithin{equation}{section}

\usepackage{cleveref}
\crefname{lemma}{Lemma}{Lemmata}
\crefname{prop}{Proposition}{Propositions}
\crefname{thm}{Theorem}{Theorems}
\crefname{cor}{Corollary}{Corollaries}
\crefname{defn}{Definition}{Definitions}
\crefname{example}{Example}{Examples}
\crefname{rem}{Remark}{Remarks}
\crefname{ass}{Assumption}{Assumptions}
\crefname{not}{Notation}{Notation}

%Symbols
\renewcommand{\~}{\tilde}
\renewcommand{\-}{\bar}
\newcommand{\bs}{\backslash}
\newcommand{\cn}{\colon}
\newcommand{\sub}{\subset}

\newcommand{\N}{\mathbb{N}}
\newcommand{\R}{\mathbb{R}}
\newcommand{\Z}{\mathbb{Z}}
\renewcommand{\S}{\mathbb{S}}
\renewcommand{\H}{\mathbb{H}}
\newcommand{\C}{\mathbb{C}}
\newcommand{\K}{\mathbb{K}}
\newcommand{\Di}{\mathbb{D}}
\newcommand{\B}{\mathbb{B}}
\newcommand{\8}{\infty}

%Greek letters
\renewcommand{\a}{\alpha}
\renewcommand{\b}{\beta}
\newcommand{\g}{\gamma}
\renewcommand{\d}{\delta}
\newcommand{\e}{\epsilon}
\renewcommand{\k}{\kappa}
\renewcommand{\l}{\lambda}
\renewcommand{\o}{\omega}
\renewcommand{\t}{\theta}
\newcommand{\s}{\sigma}
\newcommand{\p}{\varphi}
\newcommand{\z}{\zeta}
\newcommand{\vt}{\vartheta}
\renewcommand{\O}{\Omega}
\newcommand{\D}{\Delta}
\newcommand{\G}{\Gamma}
\newcommand{\T}{\Theta}
\renewcommand{\L}{\Lambda}

%Mathcal Letters
\newcommand{\cL}{\mathcal{L}}
\newcommand{\cT}{\mathcal{T}}
\newcommand{\cA}{\mathcal{A}}
\newcommand{\cW}{\mathcal{W}}

%Mathematical operators
\newcommand{\INT}{\int_{\O}}
\newcommand{\DINT}{\int_{\d\O}}
\newcommand{\Int}{\int_{-\infty}^{\infty}}
\newcommand{\del}{\partial}

\newcommand{\inpr}[2]{\left\langle #1,#2 \right\rangle}
\newcommand{\fr}[2]{\frac{#1}{#2}}
\newcommand{\x}{\times}
\DeclareMathOperator{\Tr}{Tr}

\DeclareMathOperator{\dive}{div}
\DeclareMathOperator{\id}{id}
\DeclareMathOperator{\pr}{pr}
\DeclareMathOperator{\Diff}{Diff}
\DeclareMathOperator{\supp}{supp}
\DeclareMathOperator{\graph}{graph}
\DeclareMathOperator{\osc}{osc}
\DeclareMathOperator{\const}{const}
\DeclareMathOperator{\dist}{dist}
\DeclareMathOperator{\loc}{loc}
\DeclareMathOperator{\grad}{grad}
\DeclareMathOperator{\ric}{Ric}
\DeclareMathOperator{\Rm}{Rm}
\DeclareMathOperator{\weingarten}{\mathcal{W}}
\DeclareMathOperator{\inj}{inj}

%Environments
\newcommand{\Theo}[3]{\begin{#1}\label{#2} #3 \end{#1}}
\newcommand{\pf}[1]{\begin{proof} #1 \end{proof}}
\newcommand{\eq}[1]{\begin{equation}\begin{alignedat}{2} #1 \end{alignedat}\end{equation}}
\newcommand{\IntEq}[4]{#1&#2#3	 &\quad &\text{in}~#4,}
\newcommand{\BEq}[4]{#1&#2#3	 &\quad &\text{on}~#4}
\newcommand{\br}[1]{\left(#1\right)}

\newcommand{\abs}[1]{\left|{#1}\right|}


%Logical symbols
\newcommand{\Ra}{\Rightarrow}
\newcommand{\ra}{\rightarrow}
\newcommand{\hra}{\hookrightarrow}
\newcommand{\mt}{\mapsto}

%Notes
\newcommand{\pa}[1]{{\color{green} pa: {#1}}}
\newcommand{\jj}[1]{{\color{red} jj: {#1}}}
\newcommand{\mni}[1]{{\color{blue} mni: {#1}}}

%Fonts
\newcommand{\mc}{\mathcal}
\renewcommand{\it}{\textit}
\newcommand{\mrm}{\mathrm}

%Spacing
\newcommand{\hp}{\hphantom}


%\parindent 0 pt

\protected\def\ignorethis#1\endignorethis{}
\let\endignorethis\relax
\def\TOCstop{\addtocontents{toc}{\ignorethis}}
\def\TOCstart{\addtocontents{toc}{\endignorethis}}

\DeclareMathOperator{\E}{\mathcal{E}}
\DeclareMathOperator{\bigo}{\mathcal{O}}
\DeclareMathOperator{\littleo}{o}
\declaretheorem[name=Main Theorem]{mainthm}
\crefname{mainthm}{Main Theorem}{Main Theorems}

\begin{document}

\title[]
 {Blow up of equivariant harmonic map heat flow and Yang Mills flow}

\curraddr{}
\email{}

\dedicatory{}
\subjclass[2010]{}
\keywords{}

\begin{abstract}
\end{abstract}

\maketitle

\section{Introduction}
\label{sec:intro}

We consider the phenomena of \emph{blow up} for solutions of the problem,
\begin{equation}
\label{eq:pde}
\begin{cases}
\partial_t \psi &= \psi_{rr} + \frac{C}{r} \psi_r - \frac{1}{r^2} F(\psi) \\
\psi(r, 0) &= \psi_0(r) \\
\psi(0, t) &= 0, \quad \psi(1, t) = \psi_1, \quad t \in [0, T)
\end{cases}
\end{equation}
where \(C > 0\) is a real constant and \(F\) is a positive, odd \(C^{1,\alpha}\) function on \((-a, a)\) for some \(a \in (0, \infty]\) that is increasing on an interval \((-\delta, \delta)\) around \(0\). Note that \(F(0) = 0\) and \(F'(0) > 0\).

Such problems arise in studying equivariant \emph{harmonic map heat flow} and equivariant \emph{Yang-Mills flow}. Let \(u\) solve
\[
\begin{cases}
\frac{1}{2} (u^2)' &= F(u) \\
u(0) &= 0.
\end{cases}
\]
so that \(F(\psi) = u(\psi) u'(\psi)\) and
\[
u(z) = \sqrt{\abs{\int_0^z F(w) dw}}.
\]
The conditions \(F > 0\), \(F(0) = 0\), \(F \in C^{1\,\alpha}\) and \(F'(0) > 0\) ensure that \(u\) is at least \(C^{1,\alpha/2}\).

Define the Riemmanian warped-product manifold \((M, g)\) by
\begin{equation}
\label{eq:warped_product}
M = (0, a) \times \S^1, \quad g = dz^2 + u^2(z) d\theta^2.
\end{equation}

Going the other way, given \(u\) defining \(g\) we recover \(F(\psi) = u(\psi) u'(\psi)\). Either way, equation \eqref{eq:pde}  with \(C = 1\) is precisely the \emph{degree one, equivariant harmonic map heat flow} of maps \(\Psi: (B^2, \delta) \to (M, g)\) where \(\delta\) is the flat metric. Here degree one, equivariant means that \(\Psi\) is of the form, \(\Psi(re^{i\theta}) = (\psi(r), \theta)\). Higher degree maps, i.e. those of the form \(\Psi(re^{i\theta}) = (\psi(r), \ell \theta)\) may also be treated - they simply result in \(F \mapsto \ell^2 F\) and here we absorb the \(\ell^2\) factor into \(F\).

Let us also remark that if \(F'(0) = 1\), the metric \(g\) extends to a \(C^{1,\alpha/2}\) metric on the open disc \(D(a)\) of radius \(a\) with \(r = 0\) corresponding to the centre of the disc. Similarly for \(F'(a) = -1\) (for \(a < \infty\)). If both conditions hold, the metric extends to a metric on \(\S^2\). If either condition fails, then we still obtain a metric on \(\S^2\) but with a conical singularity at the point(s) corresponding to \(r=0, a\).

Here are two motivating examples described in \cite{MR2332425}.

\begin{example}
Let \(u = \sin(\psi)\). Then degree one harmonic map heat flow gives \(F = \tfrac{1}{2} \sin(2\psi)\) and \((M, g)\) is the round two-sphere. Degree \(\ell\) harmonic map heat flow gives \(F = \tfrac{\ell^2}{2} \sin(2\psi)\).Higher dimensional, equivariant harmonic maps \(B^n \to \S^n\) with equivariant \(\SO(n-1)\) symmetries gives rise to \(C = n-1\), \(F = \tfrac{n-1}{2} \sin(2\psi)\).
\end{example}

\begin{example}
The \(\SO(4)\) equivariant Yang-Mills flow on \(\R^4\) gives rise to \(C = 1\), \(F = 4 \psi(1 - \psi)(1 - \tfrac{1}{2}\psi)\). The higher dimensional (\(\SO(n)\) equivariant on \(\R^n\)) analogue is \(C = n-3\), \(F = 2(n-2) \psi(1-\psi)(1 - \tfrac{1}{2}\psi)\).
\end{example}

Here we are motivated by generalising some interesting properties of these two examples observed in \cite{MR2332425}. These observations are that in either example, the critical dimension (\(n=2\) for harmonic maps, \(n=4\) for Yang-Mills) in which the problems are conformally invariant gives \(C = 1\) and finite time blow up for the harmonic map heat flows (first proven in \cite{MR1180392}) and no finite time blow up for the Yang-Mills flow (first proven in \cite{MR1600272}). The curious point is that for Yang-Mills flow, the metric \(g\) has a conical singularity but no blow up, whereas for the harmonic map heat flow, the metric is smooth and we do have blow up. Evidently, it is not the conical singularity that determines blow up. The answer given is that the coefficient \(\ell^2 = 1\) for harmonic map heat flow, and \(\ell^2 = 4\) for Yang-Mills flow determines blow up.

For us this translates to letting \(\ell^2 = F'(0)\). In other words, blow up is determined by the \emph{linearisation of \(F\) about \(\psi \equiv 0\)}. Let us define,
\[
\bar{F} (\psi) = F(\psi) - F'(0) \psi
\]
and the linear operator,
\begin{equation}
\label{eq:L}
L(\psi) = \psi_{rr} + \frac{C}{r} \psi_r - \frac{1}{r^2} F'(0) \psi
\end{equation}
so that the harmonic map heat flow, equation \eqref{eq:pde} becomes
\begin{equation}
\partial_t \psi - L(\psi) = - \frac{1}{r^2} \bar{F}(\psi).
\end{equation}

The operator \(L\) is an Euler operator with \emph{characteristic equation}
\begin{equation}
\label{eq:char_eqn}
p(\beta) = \beta^2 + (C-1) \beta - F'(0) = 0.
\end{equation}
The discriminant is
\[
\Delta = (C-1)^2 + 4 F'(0) > 0,
\]
so we always have real, distinct roots
\[
\frac{1 - C}{2} \pm \frac{1}{2} \sqrt{(C-1)^2 + 4 F'(0)}.
\]
Notice that since \(F'(0) > 0\), we have
\[
\frac{1}{2} \sqrt{(C-1)^2 + 4 F'(0)} > \left|\frac{1-C}{2}\right|
\]
and hence one root is positive while the other is negative. Solutions of \(L(\psi) = 0\) are of the form \(\psi(r) = r^{\beta^{\pm}}\) where \(\beta^{\pm}\) are the roots of the characteristic equation \eqref{eq:char_eqn}. The negative root is of no interest since \(r^{\beta_-}\) blows up at \(r = 0\). For the positive root, \(r^{\beta_+}\) equals \(0\) at \(r = 0\) and has finite first derivative precisely when \(r^{\beta_+} \geq 1\).

This behaviour carries over to the flow \eqref{eq:pde} as described by our main theorem.

\begin{thm}
\label{thm:main}
Let \(\psi\) be a finite energy solution of the harmonic map heat flow equation \eqref{eq:pde}. Then \(\psi\) has \(C^1\) blow up in finite time if and only if \(\beta_+ \in (0, 1]\).
\end{thm}

\begin{rem}
We recover the results of \cite{MR2332425} by observing that when \(C = 1\), the characteristic equation becomes
\[
p(\beta) = \beta^2 - F'(0)
\]
with roots \(\beta_{\pm} = \pm \sqrt{F'(0)}\). Degree one harmonic map heat flow has \(\beta_+ = 1\) and blow up occurs. More generally, degree \(\ell\) harmonic map heat flow has \(\beta_+ = \ell\) which does not blow up for \(\ell \geq 2\) and finally, Yang-Mills has \(\beta_+ = 2\) once more without blow up.
\end{rem}

The method of proof - as used by many authors \cite{MR2332425,MR1180392} - proceeds by suitable construction of sub and super solutions either forcing or preventing blow up. These are obtained from one-parameter families of stationary solutions \(\xi = \xi_{\lambda}\) satisfying \(L(\xi) = -\tfrac{1}{r^2} \bar{F} (\xi)\) and \(\xi(0) = 0\). Since \(L\) is second order, this leaves the free parameter \(\lambda\) specifying \(\xi'(0)\). By allowing this parameter to depend on \(t\) in just the right way, we obtain the desired sub and super solutions, \(\xi_{\lambda(t)} (r)\). Blow up then occurs (or not) according to the variation of \(\xi\) with respect to \(\lambda\) and whether \(\lambda\) blows up in finite time or not.

The reason the linearised operator \(L\) determines blow up is because of standard local, parabolic estimates that show blow up, if it occurs, must occur at the pole \(r = 0\) (see Lemma \ref{lem:apriori_bounds}) coupled with the assumptions on \(F\) ensuring that \(\bar{F} = \mathcal{O}(\psi^{1+\alpha})\). Thus the linearisation \(F'(0)\) dominates as \(\psi \to 0\) which connects with the local estimates through boundary condition \(\psi(0) = 0\). Of course, there are technicalities encountered (as seen for instance in the \(L_{\infty}\) bounds of the main theorem \ref{thm:main}) and these will be expanded upon below.

\section{Energy and Gradient Flow}
\label{subsec:intro_energygradient}

It is standard that both the harmonic map heat flow and Yang-Mills flow are the gradient flow for Dirichlet energies,
\[
E(\Psi) = \frac{1}{2} \int_M \|\nabla \Psi\|^2 d\mu
\]
where \(\|\cdot\|\) and \(\nabla\) are suitably interpreted. Thus our equation \eqref{eq:pde} should be the gradient flow of energy amongst equivariant solutions. Let us record here the simple statement of this fact and make some basic observations about finite energy solutions.

\begin{defn}
\label{defn:energy}
The energy, \(\E(\psi)\) of a map \(\psi: [0, 1] \to \R\) is defined to be
\[
\E(\psi) = \frac{1}{2} \int_0^1 \left(\psi_r^2 + \frac{1}{r^2} u^2(\psi)\right) r dr.
\]
\end{defn}

\begin{lemma}
The equation \eqref{eq:pde} is the negative \(L^2\) gradient flow with respect to the weighted measure \(r dr\) for the energy \(\E\) defined in Definition \ref{defn:energy}.
\end{lemma}

\begin{proof}
Let \(\psi(r, t)\) be a compactly supported smooth one-parameter family of smooth maps \(r \mapsto \psi(r, t)\). Here to say \(\psi\) is compactly supported means that the variation vector \(v = \partial_t|_{t=0} \psi\) is compactly supported in \((0, 1)\). That is, \(\operatorname{support} v = [a, b]\) with \(0 < a < b < a\).

Then we compute
\[
\begin{split}
\partial_t \E(\psi) &= \partial_t \frac{1}{2} \int_0^1  \left(\psi_r^2 + \frac{1}{r^2} u^2(\psi)\right) r dr \\
&= \int_0^1  \left(\psi_r \psi_{rt} + \frac{1}{r^2} u(\psi)u'(\psi)\right) r dr \\
&= \int_0^1 \left((\psi_r \psi_t r)_r - \psi_{rr} \psi_t r - \frac{1}{r} \psi_r \psi_t r + \frac{1}{r^2} F(\psi) \psi_t\right) dr \\
&= - \int_0^1 \left(\psi_{rr} + \frac{1}{r} \psi_r - \frac{1}{r^2} F(\psi)\right) \psi_t r dr.
\end{split}
\]
\end{proof}

\begin{rem}
\label{rem:finite_energy}

Finite energy functions \(\psi\) must satisfy
\[
\int_0^1 \psi_r^2 r dr < \infty \text{ and } \int_0^1 \frac{u^2(\psi(r))}{r} dr < \infty
\]
So as \(r\to 0\), \(\psi_r\) must not blow up faster than \(1/r\) and (if say, \(u\) is \(C^{1,\alpha}\)) \(\abs{\psi}\) must not only be bounded, but must also be bounded above by \(r^{\beta}\) for some \(\beta > 0\). Both conditions are thus simultaneously satisfied or not satisfied at all. That is, finite energy solutions must satisfy
\[
\lim_{r \to 0} \abs{\frac{\psi(r)}{r^{\beta}}} < \infty
\]
for some \(\beta > 0\).
\end{rem}

\section{Stationary solutions}
\label{sec:stationary}

In this section we study stationary solutions with suitable boundary data with a view to creating sub and super solutions in the next section. Our stationary solutions satisfy
\begin{equation}
\label{eq:stationary}
L(\xi) = \xi_{rr} + \frac{C}{r} \xi_r - \frac{F'(0)}{r^2} \xi = \frac{1}{r^2} \bar{F}(\xi).
\end{equation}
Writing \(u = \xi, v = \xi_r\) this is equivalent to the system
\[
\begin{pmatrix}
u' \\
v'
\end{pmatrix}
=
\begin{pmatrix}
v \\
\frac{1}{r^2} F(u) - \frac{C}{r} v
\end{pmatrix}.
\]
For each fixed \(r > 0\), the right hand side is a Lipschitz function of \((u,v)\) since \(F\) is Lipschitz, but it is not uniformly Lipschitz in \(r\). The right hand side is also continuous with respect to \(r\) for \(r > 0\) but only measureable for \(r \in [0, 1]\). The Cartheodary theorem \emph{does not} guarantee the existence of an absolutely continuous solution \((u, v)\) on \([0, \rho_1)\) for some  \(\rho_1 > 0\) since the coefficient functions \(1/r\), \(1/r^2\) while measureable, are not integrable. However, below we obtain solutions by explicitly solving the equation.

On the other hand, if we can obtain a solution \(\xi\) on \([0,\rho_1)\) for some \(\rho_1 > 0\), then this solution may be extended uniquely to \(r \in [0, 1]\) by choosing any \(\rho_2 \in (0, \rho_1)\) and solving the uniformly elliptic problem \eqref{eq:stationary} subject to the boundary conditions \(\psi(\rho_2) = \xi(\rho_2)\), \(\psi(1) = \psi_1\) on \([\rho_1, 1]\). Recall \(\psi_1\) is any real number occurring as the boundary data at \(r=1\) in \eqref{eq:pde}. The solution so obtained is uniformly \(C^{1,\alpha}\) on any interval \([\rho_2, 1]\) where \(\rho_2 > 0\) but not generally up to all of \([0, 1]\).

Finally, note that requiring finite energy solutions and imposing the initial condition \(\psi(0) = 0\) will provide us with a one-parameter family of solutions, parametrised by \(\psi'(0)\) with which to construct sub and super solutions for the blow up argument later.

\subsubsection*{Stationary solutions for \(C=1\)}

Let us first consider the case \(C = 1\) which is satisfied in critical dimension. It will be convenient to work with \(F\) rather than the linearisation. We can solve equation \eqref{eq:stationary} explicitly (in terms of \(F\)) by changing variables, \(y = e^w\) and letting \(\zeta(w) = \xi(e^w)\), whence our problem becomes
\[
\begin{cases}
\zeta'' &= e^{2w} \xi'' + e^w \xi' = F (\zeta) \\
\zeta(-\infty) &= 0.
\end{cases}
\]
Multiplying by \(\zeta'\) this becomes
\[
[(\zeta')^2]' = 2 F (\zeta) \zeta'.
\]
Integrating gives
\[
(\zeta')^2 = G(\zeta) + C_1, \quad \text{where} \quad G'(Z) = 2 F(Z).
\]
By adjusting \(C_1\) if necessary, we may assume that \(G(0) = 0\). Then, referring to definition \ref{defn:energy} and remark \ref{rem:finite_energy}, in order to a obtain finite energy solution we require that
\[
\frac{1}{2} \int_0^1 \left(\psi_r^2 + \frac{1}{r^2} u^2(\psi)\right) r dr < \infty.
\]
For the gradient term, substituting
\[
\psi(r) = \zeta(\ln r)
\]
with \((\zeta')^2 = G(\zeta) + C_1\) we get
\[
\frac{1}{2} \int_0^1 \psi_r^2 r dr  = \frac{1}{2} \int_0^1 \frac{G(\zeta(\ln r^)) + C_1}{r} dr < \infty.
\]
Now if \(C_1 \ne 0\), then since \(\zeta(-\infty) = 0\) and \(G(0) = 0\), by continuity there is an interval \([0, \rho]\) on which \(\abs{G(\zeta(\ln r))} < \abs{C_1}/2\) and consequently
\[
\frac{\abs{G(\zeta(\ln r)) + C_1}}{r} dr \geq \frac{\abs{C_1}}{2} \frac{1}{r}
\]
which is not integrable. Therefore we conclude that \(C_1 = 0\) and in fact,
\begin{equation}
\label{eq:G}
G(z) = 2 \int_0^z F (s) ds = \int_0^z (u^2)' (s) ds = u^2 (z).
\end{equation}
In particular, \(G > 0\) on \((0,a)\) and \(G(0) = G(a) = 0\).

Thus we must solve
\[
\frac{\zeta'}{\sqrt{\abs{G(\zeta)}}} = 1
\]
Notice that near \(z = 0\) we have \(F (z) = F'(0) z + \bigo(z^{1+\alpha})\) so that
\[
G'(z) = 2F(z) \Rightarrow G(z) = z^2F'(0) + \bigo(z^{2 + \alpha})
\]
and so
\[
\sqrt{\abs{G(\zeta)}} = z \sqrt{F'(0) + \bigo(z^{\alpha})} = \sqrt{F'(0)} z + \bigo(z^{1 + \alpha})
\]
is continuous (even differentiable) near \(z = 0\). In fact, explicitly,
\[
\frac{1}{\sqrt{\abs{G(z)}}} = \frac{1}{u(z)}
\]
is as regular as \(u\) is away from \(0, a\).

Now our solution satisfies
\[
H (\zeta(w)) = w + \mu, \quad \text{where} \quad H'(Z) = \frac{1}{\sqrt{\abs{G(Z)}}}.
\]
for an arbitrary constant \(\mu\). Explicitly, choosing any \(a_0 \in (0, a)\) we have
\begin{equation}
\label{eq:H}
H(z) = \int_{a_0}^z \frac{1}{u(s)} ds.
\end{equation}
Varying \(a_0\) just has the effect of varying \(\mu\).

Note that since \(u(z) = \sqrt{F'(0)} z + \bigo(z^{1 + \alpha})\) near \(z=0\) and near \(z=a\), we find that \(1/u\) is not summable near \(z = 0, a\) and hence
\[
H(0) = - \int_0^{a_0} \frac{1}{u(s)} ds = - \infty, \quad H(a) = \int_{a_0}^a \frac{1}{u(s)} ds = \infty.
\]
This is also why we cannot define \(H(z) = \int_{0}^z \frac{1}{u(s)} ds\). More importantly, we have
\[
H'(z) = \frac{1}{u(z)} > 0, \quad z \in (0, a).
\]
Thus \(H\) is strictly increasing and hence a \(C^1\)-diffeomorphism of \((0, a)\) onto \(\R\) so that \(H^{-1}\) is defined and \(C^1\).

In terms of \(\xi(r) = \xi(r) = \zeta(\ln r)\), have
\[
\xi(r) = H^{-1} (\ln r + \mu) = H^{-1} (\ln \lambda r)
\]
where \(\lambda = e^{\mu}\). So \(0 < \xi(r) < a\) for all \(r \in (0, 1)\) and so our solutions automatically have \(L^{\infty}\) bounds independent of \(\mu\). We may also use the implicit relation
\[
H (\xi(r)) =  \ln \lambda r.
\]

Either way, we derive:
\begin{align*}
\xi'(r) &= \frac{1}{r}  \sqrt{G(\xi(r))} \\
\xi''(r) &= \frac{1}{r} \frac{G'(\xi(r)) \xi'(r)}{2 \sqrt{G(\xi(r))}} - \frac{1}{r^2} \sqrt{G(\xi(r))} \\
&= \frac{1}{r^2} \left(F(\xi(r))- \sqrt{G(\xi(r))}\right).
\end{align*}
from which we easily verify that we \(\xi\) is indeed a solution.

To finish this section, let us examine how the derivatives of \(\xi\) depend on \(\lambda\) as this is, along with the comparison theorem is how we study blow up in general.

We have
\[
\begin{split}
\xi' &= \frac{1}{r}  \sqrt{G(\xi(r))} = \frac{1}{r}  \sqrt{G \circ H^{-1}(\ln \lambda r)} \\
&= \frac{1}{r} \left[H^{-1} (\ln \lambda r)  + \bigo(H^{-1} (\ln \lambda r))^{1 + \gamma}\right] \\
&= 
\end{split}
\]

{\color{red}Not sure how to best expand this. Maybe work with the integral definition of \(H\) or the implicit form of the solution?}

\subsubsection*{Example: Harmonic map heat flow in \(\S^2\)}

Here we have
\[
F(Z) = \frac{\ell^2}{2} \sin(2Z)
\]
so that
\[
G(Z) = \ell^2 \sin^2(Z)
\]
and
\[
H(Z) = \frac{1}{2} \ln \left(\frac{\cos(Z) - 1}{\cos(Z) + 1}\right)
\]

\subsubsection*{Example: Harmonic map heat flow in warped products}

Here \(F = \ell^2 uu' = \tfrac{\ell^2}{2} (u^2)'\) and so
\[
G = \ell^2 u^2.
\]

\subsubsection*{Stationary solutions for arbitrary \(C\)}

In this case, we don't have the same integrability structure as with \(C = 1\). However, we can still solve (somewhat less explicitly) by factoring \(L\) as the composition of two first order differential operators. That is we define
\begin{align*}
L_+ &= \partial_r - \frac{\beta^+ - 1}{r} = r^{\beta_+ - 1} \partial_r (r^{-(\beta_+ - 1)} \cdot) , \\
L_- &= \partial_r - \frac{\beta^-}{r} = r^{\beta_-} \partial_r (r^{-\beta_-} \cdot).
\end{align*}
Since \(\beta^{\pm}\) are the roots of the characteristic equation,
\[
\beta^2 + (C - 1) \beta - F'(0) = 0
\]
we have
\[
- \beta^+ - \beta^- = C-1, \quad \beta^+ \beta^- = -F'(0).
\]
Therefore,
\[
L_+ \circ L_- = \partial_r^2 + \frac{-(\beta_+ - 1) -\beta_1}{r} + \frac{\beta_+ \beta_-}{r^2} = L.
\]

Then our stationary solution satisfies
\[
L_+ \circ L_- \xi = \frac{1}{r^2} \bar{F}(\xi)
\]
or equivalently
\[
\xi = L_-^{-1} \circ L_+^{-1} \left(\frac{1}{r^2} \bar{F}(\xi)\right).
\]
where \(L_-^{-1}, L_+^{-1}\) are any left inverses. Note that \(L_{\pm}\) being first order, linear differential operators have non-unique left inverses, determined up to arbitrary constants \(C_{\pm}\). Explicitly,
\begin{align*}
L_+^{-1} (\varphi) &= r^{\beta_+ - 1} \int r^{-(\beta_+ - 1)} \varphi(r) dr + C_+ r^{\beta_+-1}, \\
L_-^{-1} (\varphi) &= r^{\beta_-} \int r^{-\beta_-} \varphi(r) dr + C_- r^{\beta_-}
\end{align*}
and hence
\begin{equation}
\label{eq:}
\begin{split}
\xi &= r^{\beta_-} \int \left[\rho^{-\beta_-} L_+^{-1} \left(\frac{1}{\rho^2} \bar{F}(\xi)\right)\right] d\rho + C_- r^{\beta_-} \\
&= r^{\beta_-} \int \left[\rho^{-\beta_-} \rho^{\beta_+ - 1} \int \left(\tau^{-(\beta_+ - 1)} \frac{1}{\tau^2} \bar{F}(\xi) \right)d\tau + \rho^{-\beta_-} C_+ \rho^{\beta_+-1}\right] d\rho + C_- r^{\beta_-} \\
&= r^{\beta_-} \int \left[\rho^{\beta_+ - \beta_- - 1} \int \left(\tau^{-(\beta_+ + 1)} \bar{F}(\xi) \right)d\tau\right] d\rho + \frac{C_+}{\beta_+ - \beta_-} r^{\beta_+} + C_- r^{-\beta_-}.
\end{split}
\end{equation}
Letting
\[
G(\varphi) = r^{\beta_-} \int \left[\rho^{\beta_+ - \beta_- - 1} \int \left(\tau^{-(\beta_+ + 1)} \bar{F}(\varphi) \right)d\tau\right] d\rho
\]
and absorbing \(\tfrac{1}{\beta_+ - \beta_-}\) (recall \(\beta_+ \ne \beta_-\)) into \(C_+\), we have
\[
(\Id - G) (\xi) = C_+ r^{\beta_+} + C_- r^{\beta_-}.
\]

Formally then, we obtain the representation formula,
\[
\xi = (\Id - G)^{-1} \left(C_+ r^{\beta_+} + C_- r^{\beta_-}\right).
\]
Note that in the case \(\bar{F} = 0\) we have \(G = 0\) and as expected, we recover the general solution, \(\xi = C_+ r{\beta_+} + C_- r^{\beta_-}\) to the linearised equation \(L(\xi) = 0\).

\begin{rem}
In the case, \(C = 1\), the characteristic equation is \(p(\beta) = \beta^2 - F'(0) = (\beta - \sqrt{F'(0)})(\beta + \sqrt{F'(0)})\), hence \(\beta_+ = -\beta_- = \sqrt{F'(0)}\) and
\[
L_+ = \partial_r - \frac{\beta_+ - 1}{r}, \quad L_- = \partial_r + \frac{\beta_+}{r}.
\]
It seems this difference of squares structure is what allows for the substitution \(r = e^w\) in the \(C=1\) case described above. Following the discussion there, the reader will observe that integrating to obtain \(\zeta'\) and then integrating again to obtain \(\zeta\) is nothing but factoring \(L\) as the composition of first order operators. It's just that for \(C = 1\), it can be done quite directly after changing variables whereas for the general case, we proceeded slightly less directly. Needless to say, this factoring for \(C = 1\) was the motivation for the general case.

For the sake of completeness, when \(C \ne 1\), the substitution, \(r = (1-C)^{1/(1-C)} w^{1/(1-C)}\) leads to
\[
\zeta'' = \frac{1}{(1-C)^2 w^2} F(\zeta)
\]
which is not so easy to integrate directly, not to mention the complications arising when \(C < 1\) in the coefficient \((1 - C)^{1/(1 - C)}\).
\end{rem}

{\color{red} We now need a contraction mapping to assert the existence of an inverse}.

{\color{red}We also need to determine how the solutions vary with respect to the free parameters which is needed for barrier construction}.

{\color{red}Work out asymptotics as \(r\to 0\) which is important for barrier construction}.

\section{Blow up}

Recall, our differential equation is
\[
\partial_t \psi = \psi_{rr} + \frac{C}{r} \psi_r - \frac{1}{r^2} F(\psi)
\]
Here we show that for bounded solutions, the phenomena of blow depends only on \(C\) and \(\lim_{z\to 0} z^{-1} F(z)\). Since we assume \(F\) is Lipschitz, we have \(|z^{-1} F(z)|\) is bounded near \(0\) and we now make the further assumption that the limit exists in which case, since \(F(0) = 0\) and \(F\) is odd and increasing,
\[
A = \lim_{z\to 0} z^{-1} F(z) \geq 0.
\]
The model case is if \(F\) is differentiable at \(z=0\), in which case the limit is simply \(A = F'(0)\) and if \(F\) is say \(C^2\) near \(z=0\), then \(F(z) = Az + \mathcal{O} (z^2)\) as \(z \to 0\).

\subsection{Comparison Principle}

Our argument is based on the following comparison principle:

\begin{thm}[Comparison Principle]
Let \(\psi\) satisfy equation \eqref{eq:pde} and let \(\overline{\xi}\) be a super-solution; that is
\begin{equation}
\label{eq:pde_super}
\begin{cases}
\partial_t \overline{\xi} &\geq \overline{\xi}_{rr} + \frac{C}{r} \overline{\xi}_r - \frac{1}{r^2} F(\overline{\xi}) \\
\overline{\xi}(r, 0) &\geq \psi_0(r) \\
\overline{\xi}(0, t) &= 0, \quad \overline{\xi}(1, t) \geq \psi_1, \quad t \in [0, T).
\end{cases}
\end{equation}
Then \(\psi \leq \overline{\xi}\) for all \(t \in [0, T)\). If on the other hand, \(\underline{\xi}\) satisfies the conditions of equation \eqref{eq:pde_super}, but with all inequalities reversed, then \(\psi \geq \underline{\xi}\).
\end{thm}

The proof is basically a standard comparison principle, but with a slight adjustment to deal with the singular behaviour of the coefficients \(1/r\) and \(F(\psi(r))/r^2\) near \(r=0\). Note also that the only non-linearity occurs in the lowest order term \(F(\psi(r))/r^2\). These constraints account for the assumptions on \(F\) given in the introduction (see equations \eqref{eq:near_positive} and \eqref{eq:far_positive}).

\begin{proof}
We only prove the super-solution case, \(\overline{\xi}\). The sub-solution case \(\underline{\xi}\) is similar.

Let
\[
\varphi = \psi - \overline{\xi}.
\]
Then the initial assumption is \(\varphi(r, 0) \leq 0\) and the claim of the theorem is that \(\varphi(r, t) \leq 0\) for all \(t \in [0, T)\). As usual we argue by contradiction: If the claim is false, then there is a \(\tau > 0\) such that
\begin{equation}
\label{eq:false_claim}
\sup \{\varphi(r, t) : 0 \leq r \leq 1, 0 \leq t \leq \tau\} > 0.
\end{equation}

Before going through the contradiction argument, we first need to handle the singular coefficients at \(r = 0\). Since \(\psi(0, t) = \overline{\xi}(0, t) = 0\), by the regularity of \(\psi\) and \(\overline{\xi}\), there exists a \(\rho > 0\) such that
\[
-\delta < \psi, \xi < \delta, \quad \text{for} \quad 0 \leq r \leq \rho, \quad 0 \leq t \leq \tau.
\]
By the assumptions \(F(0) = 0\), and \(F\) is odd, increasing on \((-\delta,\delta)\), we must have
\begin{equation}
\label{eq:near_positive}
0 \leq \frac{F(\psi) - F(\overline{\xi})}{\psi - \overline{\xi}}, \quad (r, t) \in [0, \rho] \times [0, \tau].
\end{equation}

Next, to handle \(r\) away from \(\rho\), we choose any \(\lambda\) such that
\[
\lambda > \frac{1}{\rho^2} \operatorname{Lip} (F) > 0
\]
where \(\operatorname{Lip} (F)\) is the Lipschitz constant of \(F\). Then for  \(r \geq \rho\) we have
\begin{equation}
\label{eq:far_positive}
\lambda + \frac{1}{r^2} \frac{F(\psi) - F(\overline{\xi})}{\psi - \overline{\xi}} > \lambda - \frac{1}{r^2} \operatorname{Lip} (F) > 0
\end{equation}

Now we may define
\[
h = e^{-\lambda t} \varphi.
\]
Note that \(\tau\) is already given by assuming the theorem is false in \eqref{eq:false_claim} and is independent of \(\lambda\) which is essential. A lower bound for the constant \(\lambda\) does depend on \(\tau\) through \(\rho\) but we are free to increase \(\lambda\).

We may now apply the usual contradiction argument as follows: The assumptions of the theorem give
\[
h(0, t) \leq 0, \quad h(1, t) \leq 0 \quad \text{and} \quad h(r, 0) \leq 0
\]
while claiming the conclusion of the theorem is false implies, by equation \eqref{eq:false_claim} that
\[
\sup \{h(r, t) : 0 \leq r \leq 1, 0 \leq t \leq \tau\} > 0.
\]
Let \((r_0, t_0) \in (0, 1) \times (0, \tau]\) realise the supremum. Then at \((r_0, t_0)\) we have
\begin{equation}
\label{eq:max_principle_inequality}
\begin{split}
\partial_t h &= - \lambda e^{-\lambda t} \varphi + e^{-\lambda t} \partial_t \varphi \\
&\leq  -\lambda e^{-\lambda t} \varphi + e^{-\lambda t} \left[\psi_{rr} + \frac{C}{r} \psi_r - \frac{1}{r^2} F(\psi)\right] \\
&\quad - e^{-\lambda t}\left[\overline{\xi}_{rr} + \frac{C}{r} \overline{\xi}_r - \frac{1}{r^2} F(\overline{\xi})\right] \\
&= h_{rr} + \frac{C}{r} h_r - \left(\lambda + \frac{1}{r^2} \frac{F(\psi) - F(\overline{\xi})}{\psi - \overline{\xi}}\right) h.
\end{split}
\end{equation}
Moreover, \(h(r_0, t_0)\) equals the supremum which is assumed positive, hence by equation \eqref{eq:max_principle_inequality},
\[
\lambda + \frac{1}{r^2} \frac{F(\psi) - F(\overline{\xi})}{\psi - \overline{\xi}} \leq \frac{1}{h} \left(h_{rr} + \frac{C}{r} h_r - \partial_t h\right) \leq 0.
\]
But this gives a contradiction since using equation \eqref{eq:near_positive} for \(r_0 \leq \rho\), and equation \eqref{eq:far_positive} for \(r_0 \geq \rho\) gives
\[
\lambda + \frac{1}{r^2} \frac{F(\psi) - F(\overline{\xi})}{\psi - \overline{\xi}} > 0.
\]
\end{proof}

\subsection{Blow up at the origin}
\label{subsec:origin_blowup}

With our heuristic motivation to hand, we move on to show that blow up of bounded solutions, if it occurs at all, must occur at the origin. This is a straight forward application of standard estimates \cite[Theorem 10.1]{Ladyzhenskaja:/1967} but we record the details in the following lemma for convenience.

\begin{lemma}
\label{lem:apriori_bounds}
Let \(\psi\) be a solution of \eqref{eq:pde} with initial data \(\psi_0\) satisfying
\[
\|\psi_0\|_{C^{2,\alpha}([0, 1])} < \infty,
\]
and
\[
\|\psi\|_{L^{\infty} ([0, 1] \times [0, \tau])} < \infty
\]
If there exists a \(\rho \in (0, 1]\) and a \(\tau \in (0, T)\) such that
\[
\|\psi\|_{C^{2,\alpha}([0, \rho] \times [0, \tau])} < \infty,
\]
then
\[
\|\psi\|_{C^{2,\alpha}([0, 1] \times [0, \tau])} < \infty.
\]
In particular, if the maximal existence time \(T < \infty\), then
\[
\lim_{\tau\to T} \inf_{\rho \in (0, 1)} \{\|\psi\|_{C^2([0, \rho] \times [0, \tau])}\} = \infty.
\]
\end{lemma}

\begin{proof}
We need to show
\[
\|\psi\|_{C^{2,\alpha}([\rho, 1] \times [0, \tau])} < \infty,
\]
under the hypothesis
\[
\|\psi\|_{C^{2,\alpha}([0, \rho] \times [0, \tau])} < \infty.
\]

We use the notation of \cite[Theorem 10.1]{Ladyzhenskaja:/1967}. Define the linear operator
\[
\mathcal{L} \psi = \partial_t \psi - \psi_{rr} - \frac{C}{r} \psi_r
\]
and
\[
f(r, t) = -\frac{1}{r^2} F(\psi(r, t))
\]
so that
\[
\mathcal{L} \psi = f.
\]

Let \(\Omega = (0, 1)\), \(\Omega' = (\rho, 1)\), and \(\Omega'' = (\rho/2, 1)\), \(S = \{0, 1\}\), \(S' = S'' = \{1\}\). \cite[Theorem 10.1]{Ladyzhenskaja:/1967} gives constants \(c_1, c_2 > 0\) such that
\[
\begin{split}
\|\psi\|_{C^{2,\alpha}([\rho, 1] \times [0, \tau])} &\leq c_1 \left(\|\tfrac{1}{r^2} F(\psi)\|_{C^{0,\alpha}([\rho/2, 1] \times [0, \tau])} + \|\psi_0\|_{C^{2,\alpha}([\rho/2, 1])} + \|\psi\|_{C^{2,\alpha}(\{1\} \times [0, \tau])} \right) \\
&\quad + c_2 \|\psi\|_{C([\rho/2, 1] \times [0, \tau])}.
\end{split}
\]

In the hypothesis of the theorem, we assume \(\psi\) has finite \(C^{0,\alpha}\) norm, hence the last term is finite. Since \(F\) is also Lipschitz, the first term \(r^{-2} F(\psi(r, t))\) has finite \(C^{0,\alpha}\) norm on \(r \in [\rho, 1], t \in [0, \tau]\). The second term is the initial data \(\psi_0\) which is assumed to have finite \(C^{2,\alpha}\) norm and finally the third term - the boundary term \(\psi|_{r=1}\) - is constant hence also has finite norm.
\end{proof}

\subsection{Barriers}
\label{subsec:barriers}

The argument in this section is based on constructing a one-parameter family of stationary-solutions from the stationary solutions constructed in the last section.

\subsubsection{The case \(\|\psi\|_{\infty} < a\)}

In this case, we can just use the stationary solution as a barrier. Recall that
\[
\xi_{\lambda} (r) = H^{-1} (\ln r^{\beta} + \lambda),
\]
with \(H : (0, a) \to (-\infty, \infty)\). So we automatically have \(0 < \xi < a\), as well as
\[
\lim_{r\to 0} H^{-1} (\ln r^{\beta} + \lambda) = 0, \quad \lim_{\lambda\to\infty} H^{-1}(\ln r^{\beta} + \lambda) = a \quad \text{(non uniformly in \(r\))}.
\]

{\color{red}If \(\|\psi\|_{\infty} < a\), then we for any \(r_0 > 0\), from the monotonicity of \(H^{-1} (\ln r^{\beta} + \lambda)\) in \(r\), we can choose \(\lambda\) so that \(\psi(r, 0) < \xi_{\lambda}(r)\) for \(r \in [r_0, 1]\). I need to check the asymptotics as \(r\to 0\) to get \(\psi(r, 0) < \xi(r)\) for \(r \in (0, r_0)\).}

\subsection{The case \(\|\psi\|_{\infty} < 2a\)}.

In this case, we add together two stationary solutions \(\xi_{\lambda}\) and \(\xi_{\mu}\), allowing \(\lambda\) to depend on \(t\). That is,
\[
\xi(r) = \xi_{\lambda(t)} + \xi_{\mu}
\]
where \(\lambda\) will be chosen to ensure we get sub/super solutions as appropriate, and \(\mu > 0\) is a constant. The function \(\lambda(t)\) will have some constants \(A > 0, \lambda_0 > 0\), and we'll choose \(\mu, \lambda_0\) to get the initial inequality and \(A\) to obtain a sub/super solution.

The key thing is to understand the non-linear term \(F(u)\). In particular the second order part. So let us write
\[
F(Z) = F_1 Z + F_2(Z)
\]
with \(F_1 = F'(0) \in \R\) a constant, and \(F_2(Z) = F(Z) - F_1 Z\) a \(C^{1,\alpha}\) function such that
\[
F_2'(0) = 0, \quad \lim_{r\to 0} \frac{1}{Z} F_2(Z) = 0, \quad \lim_{Z\to 0} \frac{\abs{F_2(Z)}}{Z^{1+\alpha}} < \infty.
\]

Then we define,
\[
L(\psi) = L_F(\psi) = \psi_{rr} + \frac{1}{r} \psi_r - \frac{F_1}{r^2} \psi.
\]
Our harmonic map heat flow equation is
\[
\partial_t \psi = L(\psi) - \frac{1}{r^2} F_2(\psi)
\]
and stationary solutions satisfy
\[
L(\psi) = \frac{1}{r^2} F(\psi).
\]

Our candiate barrier, \(\xi(r) = \xi_{\lambda(t)} + \xi_{\mu}\) satisfies
\[
\partial_t \xi = \partial_{\lambda} \xi_{\lambda} \partial_t \lambda
\]
and
\[
\begin{split}
L(\psi) - \frac{1}{r^2} F(\psi) &= L(\xi_{\lambda} + \xi_{\mu}) - \frac{1}{r^2} F_2(\xi_{\lambda} + \xi_{\mu}) \\
&= L(\xi_{\lambda}) + L(\xi_{\mu}) - \frac{1}{r^2} F_2(\xi_{\lambda} + \xi_{\mu}) \\
&= \frac{1}{r^2}\left[F_2(\xi_{\lambda}) + F_2(\xi_{\mu}) - F_2(\xi_{\lambda} + \xi_{\mu})\right].
\end{split}
\]

Thus we obtain a sub solution if and only if
\[
\partial_{\lambda} \xi_{\lambda} \partial_t \lambda \leq \frac{1}{r^2}\left[F_2(\xi_{\lambda}) + F_2(\xi_{\mu}) - F_2(\xi_{\lambda} + \xi_{\mu})\right]
\]
and a super solution if and only if we have \(\geq\) instead.

\end{document}
