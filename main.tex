\documentclass{amsart}

\input{StandardPaper2.tex}

\begin{document}

\title[]
 {Blow up of equivariant harmonic map heat flow and Yang Mills flow}

\curraddr{}
\email{}

\dedicatory{}
\subjclass[2010]{}
\keywords{}

\begin{abstract}
\end{abstract}

\maketitle

\section{Introduction}
\label{sec:intro}

Consider a warped product metric of the form,
\[
ds^2 = dz^2 + u(z)^2 d\phi^2
\]
on $M = \{(z,\phi) \in (0, a) \times (0, 2\pi)\}$. The metric extends smoothly to an open disc of radius $a > 0$ in $\R^2$ provided
\[
\begin{cases}
u(0) &= 0 \\
u'(0) &= 1 \\
u(-z) &= - u(z).
\end{cases}
\]
If in addition,
\[
\begin{cases}
u(a) &= 0 \\
u'(a) &= -1 \\
\end{cases}
\]
the the metric extends smoothly to $\S^2$. If $u'(a) \ne 0$, then we still obtain a metric on $\S^2$ but with a conical singularity at $a$.

We consider maps
\[
f : B \times [0, T) \to (\S^2, ds^2)
\]
where $B \subset \R^2$ is the unit ball equipped with the flat metric. Let us write $(r, \theta)$ for polar coordinates on $\R^2$ identifying $B \backslash \{0\}$ with $((0, 1) \times (0, 2\pi), dr^2 + r^2 d\theta^2)$. For a non-zero natural number \(\ell \in \N\), we say $f$ is \emph{degree $\ell$-equivariant} if
\[
f (r, \theta, t) = (\psi(r, t), \ell\theta (\operatorname{mod} 2\pi)).
\]
for some smooth $\psi : (0, 1) \times [0, T) \to (0, a)$ with \(\psi(0, t) = 0\) for every \(t \in [0, T)\).

We are interested in the phenomena of \emph{blow-up} of the \emph{harmonic map heat flow}, for which purpose it is sufficient to fix the boundary data, \(\psi(1, t) = \psi_1\) for a real constant \(\psi_1 > 0\). Of course the initial condition for \(f_0(r, \theta) = f(r, \theta, 0)\) is also equivariant so we take initial condition, \(\psi_0\) with \(\psi_0(0) = 0\) and \(\psi_0(1) = \psi_1\).

The harmonic map heat flow with the stated boundary conditions then becomes
\begin{equation}
\label{eq:harmonic_map_heat_flow}
\begin{cases}
\partial_t \psi &= \psi_{rr} + \frac{1}{r} \psi_r - \frac{\ell^2}{r^2} F(\psi) \\
\psi(r, 0) &= \psi_0(r) \\
\psi(0, t) &= 0, \quad \psi(1, t) = \psi_1, \quad t \in [0, T)
\end{cases}
\end{equation}
where \(F(z) = u(z)u'(z)\).

\subsection*{Examples}

\subsubsection*{Harmonic maps into the sphere.}

The round metric on the sphere is
\[
ds^2 = dz^2 + \sin^2(z) d\phi^2
\]
for \(z \in (0, \pi)\) and where \(d\phi^2\) is the round metric on \(\S^{n-1}\). The corresponding harmonic map heat flow is
\[
\partial_t \psi = \psi_{rr} + \frac{n-1}{r} \psi_r - \frac{n-1}{r^2} \frac{1}{2} \sin (2\psi).
\]
The so-called \emph{critical dimension} here is \(n=2\) for which the harmonic map heat flow is conformally invariant. In this case,
\[
\partial_t \psi = \psi_{rr} + \frac{1}{r} \psi_r - \frac{1}{r^2} \frac{1}{2} \sin (2\psi)
\]
and \(F(z) = \tfrac{1}{2} \sin(2z)\) in equation \eqref{eq:harmonic_map_heat_flow}.

\subsubsection*{\(SO(4)\)-equivariant Yang-Mills flow in \(\R^4\).}

Equivariant Yang-Mills flow reduces to
\[
\partial_t \psi = \psi_{rr} + \frac{n-3}{r} \psi_r - \frac{2(n-2)}{r^2} \psi(1 - \psi)(1 - \tfrac{1}{2} \psi).
\]

Here the critical dimension is \(n=4\) for which the Yang-Mills flow is conformally invariant. In this case,
\[
\partial_t \psi = \psi_{rr} + \frac{1}{r} \psi_r - \frac{1}{r^2} 4 \psi(1 - \psi)(1 - \tfrac{1}{2} \psi)
\]
and \(F(z) = 4 z(1 - z)(1 - \tfrac{1}{2} z)\) in equation \eqref{eq:harmonic_map_heat_flow}.

\section{Comparison Principle}

Our argument is based on a comparison principle. In this section, let us consider the more general equation
\begin{equation}
\label{eq:pde}
\begin{cases}
\partial_t \psi &= \psi_{rr} + \frac{C}{r} \psi_r - \frac{1}{r^2} F(\psi) \\
\psi(r, 0) &= \psi_0(r) \\
\psi(0, t) &= 0, \quad \psi(1, t) = \psi_1, \quad t \in [0, T)
\end{cases}
\end{equation}
where \(C > 0\) is a real constant and \(F\) is a Lipschitz function that is odd and increasing on some interval \((-\delta, \delta)\) around \(0\). Note that \(F(0) = 0\) since \(F\) is odd. A particular example is any \(C^2\), odd function \(F\) such that \(F(0)\) and \(F'(0) > 0\).

\begin{thm}[Comparison Principle]
Let \(\psi\) satisfy equation \eqref{eq:pde} and let \(\overline{\xi}\) be a super-solution; that is
\begin{equation}
\label{eq:pde_super}
\begin{cases}
\partial_t \overline{\xi} &\geq \overline{\xi}_{rr} + \frac{C}{r} \overline{\xi}_r - \frac{1}{r^2} F(\overline{\xi}) \\
\overline{\xi}(r, 0) &\geq \psi_0(r) \\
\overline{\xi}(0, t) &\geq 0, \quad \overline{\xi}(1, t) \geq \psi_1, \quad t \in [0, T).
\end{cases}
\end{equation}
Then \(\psi \leq \overline{\xi}\) for all \(t \in [0, T)\). If on the other hand, \(\underline{\xi}\) satisfies the conditions of equation \eqref{eq:pde_super}, but with all inequalities reversed, then \(\psi \geq \underline{\xi}\).
\end{thm}

The proof is basically a standard comparison principle, but with a slight adjustment to deal with the singular behaviour of the coefficients \(1/r\) and \(F(\psi(r))/r^2\) near \(r=0\). Note also that the only non-linearity occurs in the lowest order term \(F(\psi(r))/r^2\). These constraints account for the assumptions on \(F\) (see equations \eqref{eq:near_positive} and \eqref{eq:far_positive}).

\begin{proof}
We only prove the super-solution case, \(\overline{\xi}\). The sub-solution case \(\underline{\xi}\) is similar.

Let \(\varphi = \psi - \overline{\xi}\). Then the initial assumption is \(\varphi(r, 0) \leq 0\) and the claim of the theorem is that \(\varphi(r, t) \leq 0\) for all \(t \in [0, T)\). As usual we argue by contradiction: If the claim is false, then there is a \(\tau > 0\) such that
\begin{equation}
\label{eq:false_claim}
\sup \{\varphi(r, t) : 0 \leq r \leq 1, 0 \leq t \leq \tau\} > 0.
\end{equation}

Since \(\psi(0, t) = \overline{\xi}(0, t) = 0\), by the regularity of \(\psi\) and \(\overline{\xi}\), there exists a \(\rho > 0\) such that
\[
-\delta < \psi, \xi < \delta
\]
for \(0 \leq r \leq \rho\) and \(0 \leq t \leq \tau\). By the assumptions \(F(0) = 0\), and \(F\) is odd, increasing on \((-\delta,\delta)\), we must have
\begin{equation}
\label{eq:near_positive}
0 \leq \frac{F(\psi) - F(\overline{\xi})}{\psi - \overline{\xi}}, \quad (r, t) \in [0, \rho] \times [0, \tau].
\end{equation}
Choose any
\[
\lambda > \frac{1}{\rho^2} \operatorname{Lip} (F) > 0
\]
where \(\operatorname{Lip} (F)\) is the Lipschitz constant of \(F\). Then for  \(r \geq \rho\) we have
\begin{equation}
\label{eq:far_positive}
\lambda + \frac{1}{r^2} \frac{F(\psi) - F(\overline{\xi})}{\psi - \overline{\xi}} > \lambda - \frac{1}{r^2} \operatorname{Lip} (F) > 0
\end{equation}

Now we may define
\[
h = e^{-\lambda t} \varphi.
\]
and apply the usual contradiction argument as follows: The assumptions of the theorem give
\[
h(0, t) \leq 0, \quad h(1, t) \leq 0 \quad \text{and} \quad h(r, 0) \leq 0
\]
while claiming the conclusion of the theorem is false implies, by equation \eqref{eq:false_claim}, that
\[
\sup \{h(r, t) : 0 \leq r \leq 1, 0 \leq t \leq \tau\} > 0.
\]
Let \((r_0, t_0) \in (0, 1) \times (0, \tau]\) realise the supremum. Computing, we see that \(h\) satisfies
\[
\begin{split}
\partial_t h &= - \lambda e^{-\lambda t} \varphi + e^{-\lambda t} \partial_t \varphi \\
&\leq  -\lambda e^{-\lambda t} \varphi + e^{-\lambda t} \left[\psi_{rr} + \frac{C}{r} \psi_r - \frac{1}{r^2} F(\psi)\right] \\
&\quad - e^{-\lambda t}\left[\overline{\xi}_{rr} + \frac{C}{r} \overline{\xi}_r - \frac{1}{r^2} F(\overline{\xi})\right] \\
&= h_{rr} + \frac{C}{r} h_r - \left(\lambda + \frac{1}{r^2} \frac{F(\psi) - F(\overline{\xi})}{\psi - \overline{\xi}}\right) h.
\end{split}
\]
At \((r_0, t_0)\) realising the supremum, \(h(r_0, t_0) > 0\), and hence
\[
\lambda + \frac{1}{r^2} \frac{F(\psi) - F(\overline{\xi})}{\psi - \overline{\xi}} \leq \frac{1}{h} \left(h_{rr} + \frac{C}{r} h_r - \partial_t h\right) \leq 0.
\]
But this gives a contradiction since using equation \eqref{eq:near_positive} for \(r_0 \leq \rho\), and equation \eqref{eq:far_positive} for \(r_0 \geq \rho\) gives
\[
\lambda + \frac{1}{r^2} \frac{F(\psi) - F(\overline{\xi})}{\psi - \overline{\xi}} > 0.
\]
\end{proof}

\section{Blow up}

Our equation is
\[
\partial_t \psi = \psi_{rr} + \frac{C}{r} \psi_r - \frac{1}{r^2} F(\psi)
\]
Here we show that for bounded solutions, the phenomena of blow depends only on \(C\) and \(\lim_{z\to 0} z^{-1} F(z)\). Since we assume \(F\) is Lipschitz, we have \(|z^{-1} F(z)|\) is bounded near \(0\) and we now make the further assumption that the limit exists in which case, since \(F(0) = 0\) and \(F\) is odd and increasing,
\[
A = \lim_{z\to 0} z^{-1} F(z) \geq 0.
\]
The model case is if \(F\) is differentiable at \(z=0\), in which case the limit is simply \(A = F'(0)\) and if \(F\) is say \(C^2\), then \(F(z) = Az + \mathcal{O} (z^2)\) as \(z \to 0\).

\subsection*{Heuristic Analysis}

Because the coefficients are singular at \(r=0\), finite time blow up, if it occurs, must occur at the origin \(r = 0\) (see Lemma \ref{lem:apriori_bounds}). Heuristically then, to see when blow up occurs we need only consider the asymptotics near \(r = 0\). For the purpose of illustration then, it's sufficient to take \(\psi = r^{\beta}\) and \(F(z) = A z\) with \(A > 0\).

With our choice of \(\psi\),
\[
\partial_t \psi = 0
\]
while
\[
\psi_{rr} + \frac{C}{r} \psi_r - \frac{1}{r^2} F(\psi) = \left[\beta(\beta-1) + C \beta - A\right]r^{\beta-2}.
\]
Thus \(\psi = r^{\beta}\) is a
\begin{equation}
\label{eq:betasubsuper}
\begin{cases}
\text{sub-solution if } & \beta(\beta-1) + C \beta - A \geq 0, \\
\text{solution if } & \beta(\beta-1) + C \beta - A = 0, \\
\text{super-solution if } & \beta(\beta-1) + C \beta - A \leq 0.
\end{cases}
\end{equation}
In conjunction with the comparison principle, sub-solutions are used to show blow up while super-super solutions are used to show no blow-up occurs. A solution may of course be used for either purpose.

Recall our boundary condition \(\psi(0, t) = 0\) which rules out the usefulness of \(\beta \leq 0\). The range \(0 < \beta < 1\) models \(C^1\) blow up while \(1 \leq \beta\) models no \(C^1\) blow up. The range \(1 < \beta < 2\) models \(C^2\) blow up and \(\beta \geq 2\) has no \(C^2\) blow up. Since our equation is second order, we typically are only interested in up-to-\(C^2\) blow up.

Therefore, for fixed \(C > 0\) and \(A > 0\) we are interested in the roots of
\[
p(\beta) = \beta^2 + (C - 1) \beta - A.
\]
The discriminant is
\[
\Delta = (C-1)^2 + 4 A > 0
\]
and so we always have real, distinct roots \(\beta_1 < \beta_2\) given by
\[
\frac{1 - C}{2} \pm \frac{1}{2} \sqrt{(C-1)^2 + 4 A}.
\]
From \eqref{eq:betasubsuper}, we have \(p(\beta) < 0\) for \(\beta_1 < \beta < \beta_2\) corresponding to super-solutions, whereas \(p(\beta) > 0\) for \(\beta < \beta_2\) and for \(\beta > \beta_1\) corresponding to sub-solutions. Then large \(\beta\) always produces sub-solutions with no blow up but this has no significance. If \(\beta_2 < 1\), then we find a range of \(\beta \in (0, 1)\) giving blowing up sub-solutions. If \(\beta_1 > 1\), then we find a range \(\beta > 1\) giving super-solutions with out blow up.

It's particularly instructive to consider the critical cases of two-dimensional harmonic map heat flow and four-dimensional Yang-Mills flow. In both cases \(C = 1\). In higher dimensions, \(C > 1\). Thus in the critical dimension,
\[
p(\beta) = \beta^2 - A
\]
hence \(\beta_1 = - \sqrt{A}\) and \(\beta_2 = \sqrt{A}\).

For the degree \(\ell\)-harmonic map heat flow, \(A = \ell^2\) while for the Yang-Mills flow, \(A = 4\). Then for degree \(1\)-harmonic map heat flow, \(\beta_2 = 1\) lies on the boundary of \(C^1\) blow up, and it turns out this may be perturbed by a time-dependent term to induce blow up. For degree \(\ell\) harmonic map heat flow and for Yang Mills flow \(\beta_2 = 2\) and no \(C^1\) blow up occurs but again a time-dependent perturbation induces \(C^2\) blow up. Higher degree harmonic map heat flow does not undergo \(C^2\) blow up. Full details are given below.

\subsection*{Blow up at the origin}

With our heuristic motivation to hand, we move on to show that blow up, if it occurs at all, must occur at the origin. This is a straight forward application of standard estimates \cite[Theorem 10.1]{Ladyzhenskaja:/1967} but we record the details in the following lemma for convenience.

\begin{lemma}
\label{lem:apriori_bounds}
Let \(\psi\) be a solution of \eqref{eq:pde}. If there exists a \(\rho \in (0, 1]\) and a \(\tau \in (0, T)\) such that
\[
\|\psi\|_{C^{k,\alpha}([0, \rho] \times [0, \tau])} < \infty,
\]
then
\[
\|\psi\|_{C^{k,\alpha}([0, 1] \times [0, \tau])} < \infty.
\]
In particular, if the maximal existence time \(T < \infty\), then
\[
\lim_{\tau\to T} \inf_{\rho \in (0, 1)} \{\|\psi\|_{C^2([0, \rho] \times [0, \tau])}\} = \infty.
\]
\end{lemma}

\begin{proof}
We need to show
\[
\|\psi\|_{C^{k,\alpha}([\rho, 1] \times [0, \tau])} < \infty,
\]
under the hypothesis
\[
\|\psi\|_{C^{k,\alpha}([0, \rho] \times [0, \tau])} < \infty.
\]

We use the notation of \cite[Theorem 10.1]{Ladyzhenskaja:/1967}. Define the linear operator
\[
\mathcal{L} \psi = \partial_t \psi - \psi_{rr} - \frac{C}{r} \psi_r
\]
and
\[
f(r, t) = -\frac{1}{r^2} F(\psi(r, t))
\]
so that
\[
\mathcal{L} \psi = f.
\]

Let \(\Omega = (0, 1)\), \(\Omega' = (\rho, 1)\), and \(\Omega'' = (\rho/2, 1)\), \(S = \{0, 1\}\), \(S' = S'' = \{1\}\). \cite[Theorem 10.1]{Ladyzhenskaja:/1967} gives constants \(c_1, c_2 > 0\) such that
\[
\begin{split}
\|\psi\|_{C^{k,\alpha}([\rho, 1] \times [0, \tau])} &\leq c_1 \left(\|\tfrac{1}{r^2} F(\psi)\|_{C^{k-2,\alpha}([\rho/2, 1] \times [0, \tau])} + \|\psi_0\|_{C^{k,\alpha}([\rho/2, 1])} + \|\psi\|_{C^{k,\alpha}(\{1\} \times [0, \tau])} \right) \\
&\quad + c_2 \|\psi\|_{C([\rho/2, 1] \times [0, \tau])}.
\end{split}
\]
\end{proof}

\subsection*{Finite time blow up}

In this section we control arbitrary solutions with suitable boundary data by sub- and super-solutions thereby either demonstrating or ruling out finite time blow up.

\end{document}
