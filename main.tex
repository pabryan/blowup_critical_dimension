\documentclass{amsart}

\input{StandardPaper2.tex}
\DeclareMathOperator{\E}{\mathcal{E}}
\DeclareMathOperator{\bigo}{\mathcal{O}}
\DeclareMathOperator{\littleo}{o}
\declaretheorem[name=Main Theorem]{mainthm}
\crefname{mainthm}{Main Theorem}{Main Theorems}

\begin{document}

\title[]
 {Blow up of equivariant harmonic map heat flow and Yang Mills flow}

\curraddr{}
\email{}

\dedicatory{}
\subjclass[2010]{}
\keywords{}

\begin{abstract}
\end{abstract}

\maketitle

\section{Introduction}
\label{sec:intro}

We consider the phenomena of \emph{blow up} for solutions of the problem,
\begin{equation}
\label{eq:pde}
\begin{cases}
\partial_t \psi &= \psi_{rr} + \frac{C}{r} \psi_r - \frac{1}{r^2} F(\psi) \\
\psi(r, 0) &= \psi_0(r) \\
\psi(0, t) &= 0, \quad \psi(1, t) = \psi_1, \quad t \in [0, T)
\end{cases}
\end{equation}
where \(C > 0\) is a real constant and \(F\) is a Lipschitz function that is odd and increasing on some interval \((-\delta, \delta)\) around \(0\). Note that \(F(0) = 0\) since \(F\) is odd. A particular example is any \(C^2\), odd function \(F\) such that \(F'(0) > 0\).

Such problems arise in studying equivariant \emph{harmonic map heat flow} and equivariant \emph{Yang-Mills flow}.

\subsection{Energy and Gradient Flow}

Let \(u\) solve
\[
\begin{cases}
\frac{1}{2} (u^2)' &= F(u) \\
u(0) &= 0.
\end{cases}
\]
Note that \(F(\psi) = u(\psi) u'(\psi)\). Explicitly we have
\[
u(z) = \sqrt{\abs{\int_0^z F(w) dw}}
\]
and the condition that \(F(0) = 0\) and \(F\) is Lipschitz increasing near \(0\) should give some regularity of \(u\) near \(z = 0\). For example if \(F\) is \(C^{1,\alpha}\) then
\[
\int_0^z F(w) dw = \int_0^z F'(0) w + \bigo(w^{1+\alpha}) dw = z^2\left(\frac{1}{2} F'(0) + \bigo(z^{\alpha})\right)
\]
so that after taking the square root, we get \(u\) is at least \(C^{1,\alpha/2}\).

Let
\[
a = \sup\{z > 0: u > 0 \text{ on } (0, z)\}
\]
where we allow \(a = \infty\). Then define the warped-product metric,
\[
g = dz^2 + u^2(z) d\theta^2
\]
on \(M = (0, a) \times \S^1\). Then an equivariant harmonic map \(\psi: (B, dr^2 + r^2 d\theta^2) \to (M, g)\) is harmonic if and only if equation \eqref{eq:pde} holds.

\begin{defn}
\label{defn:energy}
The energy, \(\E(\psi)\) of a map \(\psi: [0, 1] \to \R\) is defined to be
\[
\E(\psi) = \frac{1}{2} \int_0^1 \left(\psi_r^2 + \frac{1}{r^2} u^2(\psi)\right) r dr.
\]
\end{defn}

\begin{lemma}
The equation \eqref{eq:pde} is the negative \(L^2\) gradient flow with respect to the weighted measure \(r dr\) for the energy \(\E\) defined in Definition \ref{defn:energy}.
\end{lemma}

\begin{proof}
Let \(\psi(r, t)\) be a compactly supported smooth one-parameter family of smooth maps \(r \mapsto \psi(r, t)\). Here to say \(\psi\) is compactly supported means that the variation vector \(v = \partial_t|_{t=0} \psi\) is compactly supported in \((0, 1)\). That is, \(\operatorname{support} v = [a, b]\) with \(0 < a < b < a\).

Then we compute
\[
\begin{split}
\partial_t \E(\psi) &= \partial_t \frac{1}{2} \int_0^1  \left(\psi_r^2 + \frac{1}{r^2} u^2(\psi)\right) r dr \\
&= \int_0^1  \left(\psi_r \psi_{rt} + \frac{1}{r^2} u(\psi)u'(\psi)\right) r dr \\
&= \int_0^1 \left((\psi_r \psi_t r)_r - \psi_{rr} \psi_t r - \frac{1}{r} \psi_r \psi_t r + \frac{1}{r^2} F(\psi) \psi_t\right) dr \\
&= - \int_0^1 \left(\psi_{rr} + \frac{1}{r} \psi_r - \frac{1}{r^2} F(\psi)\right) \psi_t r dr.
\end{split}
\]
\end{proof}

\begin{rem}
\label{rem:finite_energy}

Finite energy functions \(\psi\) must satisfy
\[
\int_0^1 \psi_r^2 r dr < \infty \text{ and } \int_0^1 \frac{u^2(\psi(r))}{r} dr < \infty
\]
So as \(r\to 0\), \(\psi_r\) must not blow up faster than \(1/r\) and (if say, \(u\) is \(C^{1,\alpha}\)) \(\abs{\psi}\) must not only be bounded, but must also be bounded above by \(r^{\beta}\) for some \(\beta > 0\). Both conditions are thus simultaneously satisfied or not satisfied at all. That is, finite energy solutions must satisfy
\[
\lim_{r \to 0} \abs{\frac{\psi(r)}{r^{\beta}}} < \infty
\]
for some \(\beta > 0\).
\end{rem}

\subsection{Examples}

\subsubsection*{Harmonic maps into the round sphere.}

The round metric on the sphere \(\S^n\), expressed in polar coordinates \(\S^n \backslash \{\pm p\} \simeq (0, \pi) \times \S^{n-1}\) may be written
\[
ds^2 = dz^2 + \sin^2(z) d\phi^2
\]
for \(z \in (0, \pi)\) and where \(d\phi^2\) is the round metric on \(\S^{n-1}\). The Lie group \(O(n)\) acts on the \(\S^{n-1}\) factor producing a corresponding \(O(n)\)-equivariant harmonic map heat flow,
\[
\partial_t \psi = \psi_{rr} + \frac{n-1}{r} \psi_r - \frac{n-1}{r^2} \frac{1}{2} \sin (2\psi).
\]
The so-called \emph{critical dimension} here is \(n=2\) for which the harmonic map heat flow is conformally invariant. In this case,
\[
\partial_t \psi = \psi_{rr} + \frac{1}{r} \psi_r - \frac{1}{r^2} \frac{1}{2} \sin (2\psi).
\]

\subsubsection*{\(SO(n)\)-equivariant Yang-Mills flow in \(\R^n\).}

\(SO(n)\) equivariant Yang-Mills flow on \(\R^n\) reduces to
\[
\partial_t \psi = \psi_{rr} + \frac{n-3}{r} \psi_r - \frac{2(n-2)}{r^2} \psi(1 - \psi)(1 - \tfrac{1}{2} \psi).
\]

Here the critical dimension is \(n=4\) for which the Yang-Mills flow is conformally invariant. In this case,
\[
\partial_t \psi = \psi_{rr} + \frac{1}{r} \psi_r - \frac{1}{r^2} 4 \psi(1 - \psi)(1 - \tfrac{1}{2} \psi).
\]

\subsubsection*{Degree \(\ell\) harmonic maps into two-dimensional warped products}

Both the above situations fit into a broader framework. Consider a warped product metric of the form,
\[
ds^2 = dz^2 + u(z)^2 d\phi^2
\]
on $M = \{(z,\phi) \in (0, a) \times (0, 2\pi)\}$ and with \(u > 0\) on \((0, a)\). The metric extends smoothly to an open disc of radius $a > 0$ in $\R^2$ provided
\[
\begin{cases}
u(0) &= 0 \\
u'(0) &= 1 \\
u(-z) &= - u(z).
\end{cases}
\]
If in addition,
\[
\begin{cases}
u(a) &= 0 \\
u'(a) &= -1 \\
\end{cases}
\]
the the metric extends smoothly to $\S^2$. If $u'(a) \ne -1$, then we still obtain a metric on $\S^2$ but with a conical singularity at $a$.

We consider maps
\[
f : B \times [0, T) \to (\S^2, ds^2)
\]
where $B \subset \R^2$ is the unit ball equipped with the flat metric. Let us write $(r, \theta)$ for polar coordinates on $\R^2$ identifying $B \backslash \{0\}$ with $((0, 1) \times (0, 2\pi), dr^2 + r^2 d\theta^2)$. For a non-zero natural number \(\ell \in \N\), we say $f$ is \emph{degree $\ell$-equivariant} if
\[
f (r, \theta, t) = (\psi(r, t), \ell\theta (\operatorname{mod} 2\pi)).
\]
for some smooth $\psi : (0, 1) \times [0, T) \to (0, a)$ with \(\psi(0, t) = 0\) for every \(t \in [0, T)\).

We are interested in the phenomena of \emph{blow-up} of the \emph{harmonic map heat flow}, for which purpose it is sufficient to fix the boundary data, \(\psi(1, t) = \psi_1\) for a real constant \(\psi_1 > 0\). Of course the initial condition for \(f_0(r, \theta) = f(r, \theta, 0)\) is also equivariant so we take initial condition, \(\psi_0\) with \(\psi_0(0) = 0\) and \(\psi_0(1) = \psi_1\).

The harmonic map heat flow with the stated boundary conditions then becomes
\begin{equation}
\label{eq:harmonic_map_heat_flow}
\begin{cases}
\partial_t \psi &= \psi_{rr} + \frac{1}{r} \psi_r - \frac{\ell^2}{r^2} F(\psi) \\
\psi(r, 0) &= \psi_0(r) \\
\psi(0, t) &= 0, \quad \psi(1, t) = \psi_1, \quad t \in [0, T)
\end{cases}
\end{equation}
where \(F(z) = u(z)u'(z)\).

Note that the assumptions on \(u\) in order that we obtain a warped product metric imply that
\[
F(0) = 0, \quad F'(0) = 1,
\]
and if the metric extends to a smooth metric on \(\S^2\),
\[
F(a) = 0, \quad F'(a) = 1.
\]
In the presence of a conical singularity, \(F'(a) = (u'(a))^2 \ne 1\).

\subsubsection*{Degree \(\ell\) harmonic maps into warped products}

\textbf{todo}

Should be similar to the two-dimensional case but I don't know if a degree \(\ell\) map \(\S^{n-1} \to \S^{n-1}\) produces a factor like \(\ell^2\) as in the two-dimensional case.

\subsubsection*{Heat flow with non-linear reaction}

\(O(n)\) invariant solutions to equations of the form
\[
\partial_t u = \Delta u + F(u)
\]
on the open, unit ball fits into the framework. In fact, we can interpret such equations as Harmonic maps into a warped product space with metric
\[
ds^2 = dr^2 + g^2(r) d\theta^2
\]
where \(g\) solves the ODE
\[
\begin{cases}
gg' &= F(g) \\
g(0) &= 0
\end{cases}
\]
which has a unique solution provided \(F(g)/g\) is Lipschitz.

\textbf{Write down the energy here.}

\textbf{One can probably add a gradient term here too.}

In particular, radial solutions \(u(x, t) = \psi(|x|, t)\) of
\[
\partial_t u = \Delta u - A u^p, \quad p = 1, p \geq 2
\]
satisfy
\[
\partial_t \psi = \psi_{rr} + \frac{n-1}{r} \psi_r - \frac{1}{r^2} A u^p.
\]

\textbf{Check how the conformal factor in the Yamabe problem fits here. It seems more non-linear. Does one just mess about with \(v = u^{(n+2)/(n-2)}\) to push the non-linearity into lower order terms? Does that make the gradient term non-linear though?}

\textbf{Really we want \(\Delta u + u^p\), or in other words \(A<0\) so \(F = A u^p\) is decreasing in this case. Probably it's okay, but might need to take that into account. Perhaps best to look at equations with \(+ \frac{1}{r} F(\psi)\) and assume \(F\) is odd and either increasing or decreasing. Should work out?}

\subsubsection*{General Equivariant Harmonic Map Heat Flow}

Let \(G\) be a Lie group and \(M, N\) homogeneous \(G\) spaces. Then look at warped product metrics
\[
dr^2 + u^2(r) g
\]
where \(g\) is a left/right/bi-invariant metric on \(M\). Likewise for \(N\). Then \(G\) equivariant harmonic map heat flow should reduce to an ODE similar to the spherical case.

\subsubsection*{General Equivariant Yang-Mills Flow}

Again \(G\)-equivariance on a warped-product should reduce to an ODE.

\subsubsection*{Relationships among the examples}

The examples of harmonic maps into spheres and Yang-Mills flow are particular cases of \(F\) in \eqref{eq:harmonic_map_heat_flow}. It's worth pointing out that in the critical dimenstion, the coeefficient of \(\psi_r\) is \(1\). For the problem \eqref{eq:pde} we allow the more general situation where \(C \ne 1\) and so our study is applicible in higher dimensions. It also turns out that the value of \(F'(0)\) is crticial to the study of finite time blow up. We summarise these examples in the following table (absorbing the \(\ell^2\) into \(F\)):

\begin{center}
\begin{tabular}{ l | c | c | c}
Flow & C & \(F(z)\) & \(F'(0)\) \\
\hline
Harmonic map heat flow into \(\S^n\) & \(n-1\) & \(\tfrac{n-1}{2} \sin(2z)\) & n-1 \\
Yang-Mills flow on \(\R^n\) & \(n-3\) & \(2(n-2) z(1 - z)(1 - \tfrac{1}{2} z)\) & 2(n-2) \\
Critical dimension Harmonic map heat flow into \(\S^2\) & \(1\) & \(\tfrac{1}{2} \sin(2z)\) & 1 \\
Critical Yang-Mills flow on \(\R^4\) & \(1\) & \(4 z(1 - z)(1 - \tfrac{1}{2} z)\) & 4 \\
Degree \(\ell\) Harmonic map heat flow into \(\S^2\) & \(1\) & \(\tfrac{\ell^2}{2} \sin(2z)\) & \(\ell^2\)
\end{tabular}
\end{center}

Of particular note is the last three rows where \(C = 1\). In that case, the last column determines blow up which occurs if and only if \(F'(0) = 1\). The first two rows all have \(C > 1\) in which case blow up is dependent on \textbf{Find out what exactly!}.


\section{Comparison Principle}

Our argument is based on the following comparison principle:

\begin{thm}[Comparison Principle]
Let \(\psi\) satisfy equation \eqref{eq:pde} and let \(\overline{\xi}\) be a super-solution; that is
\begin{equation}
\label{eq:pde_super}
\begin{cases}
\partial_t \overline{\xi} &\geq \overline{\xi}_{rr} + \frac{C}{r} \overline{\xi}_r - \frac{1}{r^2} F(\overline{\xi}) \\
\overline{\xi}(r, 0) &\geq \psi_0(r) \\
\overline{\xi}(0, t) &= 0, \quad \overline{\xi}(1, t) \geq \psi_1, \quad t \in [0, T).
\end{cases}
\end{equation}
Then \(\psi \leq \overline{\xi}\) for all \(t \in [0, T)\). If on the other hand, \(\underline{\xi}\) satisfies the conditions of equation \eqref{eq:pde_super}, but with all inequalities reversed, then \(\psi \geq \underline{\xi}\).
\end{thm}

The proof is basically a standard comparison principle, but with a slight adjustment to deal with the singular behaviour of the coefficients \(1/r\) and \(F(\psi(r))/r^2\) near \(r=0\). Note also that the only non-linearity occurs in the lowest order term \(F(\psi(r))/r^2\). These constraints account for the assumptions on \(F\) given in the introduction (see equations \eqref{eq:near_positive} and \eqref{eq:far_positive}).

\begin{proof}
We only prove the super-solution case, \(\overline{\xi}\). The sub-solution case \(\underline{\xi}\) is similar.

Let
\[
\varphi = \psi - \overline{\xi}.
\]
Then the initial assumption is \(\varphi(r, 0) \leq 0\) and the claim of the theorem is that \(\varphi(r, t) \leq 0\) for all \(t \in [0, T)\). As usual we argue by contradiction: If the claim is false, then there is a \(\tau > 0\) such that
\begin{equation}
\label{eq:false_claim}
\sup \{\varphi(r, t) : 0 \leq r \leq 1, 0 \leq t \leq \tau\} > 0.
\end{equation}

Before going through the contradiction argument, we first need to handle the singular coefficients at \(r = 0\). Since \(\psi(0, t) = \overline{\xi}(0, t) = 0\), by the regularity of \(\psi\) and \(\overline{\xi}\), there exists a \(\rho > 0\) such that
\[
-\delta < \psi, \xi < \delta, \quad \text{for} \quad 0 \leq r \leq \rho, \quad 0 \leq t \leq \tau.
\]
By the assumptions \(F(0) = 0\), and \(F\) is odd, increasing on \((-\delta,\delta)\), we must have
\begin{equation}
\label{eq:near_positive}
0 \leq \frac{F(\psi) - F(\overline{\xi})}{\psi - \overline{\xi}}, \quad (r, t) \in [0, \rho] \times [0, \tau].
\end{equation}

Next, to handle \(r\) away from \(\rho\), we choose any
\[
\lambda > \frac{1}{\rho^2} \operatorname{Lip} (F) > 0
\]
where \(\operatorname{Lip} (F)\) is the Lipschitz constant of \(F\). Then for  \(r \geq \rho\) we have
\begin{equation}
\label{eq:far_positive}
\lambda + \frac{1}{r^2} \frac{F(\psi) - F(\overline{\xi})}{\psi - \overline{\xi}} > \lambda - \frac{1}{r^2} \operatorname{Lip} (F) > 0
\end{equation}

Now we may define
\[
h = e^{-\lambda t} \varphi.
\]
Note that \(\tau\) is given by assuming the theorem is false in \eqref{eq:false_claim} and \(\lambda\) depends on \(\tau\) through \(\rho\). We may now apply the usual contradiction argument as follows: The assumptions of the theorem give
\[
h(0, t) \leq 0, \quad h(1, t) \leq 0 \quad \text{and} \quad h(r, 0) \leq 0
\]
while claiming the conclusion of the theorem is false implies, by equation \eqref{eq:false_claim} that
\[
\sup \{h(r, t) : 0 \leq r \leq 1, 0 \leq t \leq \tau\} > 0.
\]
Let \((r_0, t_0) \in (0, 1) \times (0, \tau]\) realise the supremum. Computing, we see that \(h\) satisfies
\[
\begin{split}
\partial_t h &= - \lambda e^{-\lambda t} \varphi + e^{-\lambda t} \partial_t \varphi \\
&\leq  -\lambda e^{-\lambda t} \varphi + e^{-\lambda t} \left[\psi_{rr} + \frac{C}{r} \psi_r - \frac{1}{r^2} F(\psi)\right] \\
&\quad - e^{-\lambda t}\left[\overline{\xi}_{rr} + \frac{C}{r} \overline{\xi}_r - \frac{1}{r^2} F(\overline{\xi})\right] \\
&= h_{rr} + \frac{C}{r} h_r - \left(\lambda + \frac{1}{r^2} \frac{F(\psi) - F(\overline{\xi})}{\psi - \overline{\xi}}\right) h.
\end{split}
\]
At \((r_0, t_0)\), \(h(r_0, t_0) > 0\), and hence
\[
\lambda + \frac{1}{r^2} \frac{F(\psi) - F(\overline{\xi})}{\psi - \overline{\xi}} \leq \frac{1}{h} \left(h_{rr} + \frac{C}{r} h_r - \partial_t h\right) \leq 0.
\]
But this gives a contradiction since using equation \eqref{eq:near_positive} for \(r_0 \leq \rho\), and equation \eqref{eq:far_positive} for \(r_0 \geq \rho\) gives
\[
\lambda + \frac{1}{r^2} \frac{F(\psi) - F(\overline{\xi})}{\psi - \overline{\xi}} > 0.
\]
\end{proof}

\section{Blow up}

Our equation is
\[
\partial_t \psi = \psi_{rr} + \frac{C}{r} \psi_r - \frac{1}{r^2} F(\psi)
\]
Here we show that for bounded solutions, the phenomena of blow depends only on \(C\) and \(\lim_{z\to 0} z^{-1} F(z)\). Since we assume \(F\) is Lipschitz, we have \(|z^{-1} F(z)|\) is bounded near \(0\) and we now make the further assumption that the limit exists in which case, since \(F(0) = 0\) and \(F\) is odd and increasing,
\[
A = \lim_{z\to 0} z^{-1} F(z) \geq 0.
\]
The model case is if \(F\) is differentiable at \(z=0\), in which case the limit is simply \(A = F'(0)\) and if \(F\) is say \(C^2\) near \(z=0\), then \(F(z) = Az + \mathcal{O} (z^2)\) as \(z \to 0\).

\subsection{Heuristic Analysis}

Because the coefficients are singular at \(r=0\), finite time blow up of bounded solutions, if it occurs, must occur at the origin \(r = 0\) (see Lemma \ref{lem:apriori_bounds}). Heuristically then, to see when blow up occurs we need only consider the asymptotics near \(r = 0\). For the purpose of illustration then, it's sufficient to take \(\psi = r^{\beta}\) and \(F(z) = A z\) with \(A > 0\).

With our choice of \(\psi\),
\[
\partial_t \psi = 0
\]
and the spatial part is a second order, linear Euler equation:
\[
\psi_{rr} + \frac{C}{r} \psi_r - \frac{1}{r^2} F(\psi) = \left[\beta(\beta-1) + C \beta - A\right]r^{\beta-2}.
\]
Thus \(\psi = r^{\beta}\) is a
\begin{equation}
\label{eq:betasubsuper}
\begin{cases}
\text{sub-solution if } & \beta^2 + (C - 1) \beta - A \geq 0, \\
\text{solution if } & \beta^2 + (C - 1) \beta - A = 0, \\
\text{super-solution if } & \beta^2 + (C - 1) \beta - A \leq 0.
\end{cases}
\end{equation}
In conjunction with the comparison principle, sub-solutions are used to show blow up while super-super solutions are used to show no blow-up occurs. A solution may of course be used for either purpose.

Therefore, for fixed \(C > 0\) and \(A > 0\) we are interested in the roots of the characteristic equation
\begin{equation}
\label{eq:char_poly}
p(\beta) = \beta^2 + (C - 1) \beta - A.
\end{equation}
The discriminant is
\[
\Delta = (C-1)^2 + 4 A > 0,
\]
so we always have real, distinct roots
\[
\frac{1 - C}{2} \pm \frac{1}{2} \sqrt{(C-1)^2 + 4 A}.
\]
Notice that since \(A > 0\),
\[
\frac{1}{2} \sqrt{(C-1)^2 + 4 A} > \left|\frac{1-C}{2}\right|
\]
hence one root is negative which is of no interest to us because of our boundary condition \(\psi(0, t) = 0\), and the other root is positive. Let us set
\[
\beta_0 = \frac{1 - C}{2} + \frac{1}{2} \sqrt{(C-1)^2 + 4 A} > 0.
\]
Then we have \(0 < \beta \leq \beta_0\) corresponds to super-solutions while \(\beta_0 \leq \beta\) corresponds to sub-solutions.

We have
\[
\psi = r^{\beta}, \quad \psi' = \beta r^{\beta-1}, \quad \psi'' = \beta(\beta-1) r^{\beta - 2}.
\]
Thus if \(0 < \beta_0 < 1\) we obtain sub-solutions \(r^{\beta}, \beta_0 \leq \beta < 1\) with \(C^1\) blow up at the origin. If \(\beta_0 > 1\), we obtain super-solutions \(r^{\beta}, \beta_0 \leq \beta < 1\) with no \(C^1\) blow up at the origin. Similarly \(0 < \beta_0 < 2\) gives sub-solutions with \(C^2\) blow up while \(2 < \beta_0\) gives super-solutions with no \(C^2\) blow up. It turns out the boundary cases \(\beta_0 = 1, 2\) may be perturbed by a time-dependent factor to induce blow up.

Now we look at formal asymptotics near \(r = 0\). Let us suppose that \(F\) is analytic,
\[
F(z) = \sum_{n=1}^{\infty} F_n z^n
\]
and that we have an analytic solution,
\[
\psi(r) = \sum_{n=1}^{\infty} \psi_n r^n.
\]
Then
\[
\psi' = \sum_{n=0}^{\infty} (n+1) \psi_{n+1} r^n, \quad \psi'' = \sum_{n=0}^{\infty} (n+1)(n+2) \psi_{n+2} r^n.
\]
and
\[
F(\psi) = F(\sum_{n=1}^{\infty} \psi_n r^n) = \sum_{n=1}^{\infty} a_n r^n
\]
with
\[
a_n = \sum_{k=1}^n F_k S_{n,k}(\psi_i)
\]
where
\[
S_{n,k}(\psi_i) = \sum_{\pi \in P_{n,k}} {k \choose \pi} \psi_1^{\pi_1} \cdots \psi_k^{\pi_k},
\]
and
\[
P_{n,k} = \{\pi = (\pi_1, \dots, \pi_n) : \sum_{i=1}^n \pi_i = k, \sum_{i=1}^n i \pi_i = n\},
\]
is a sum over partitions of \(n\) into \(k\) parts in frequency-of-parts form and
\[
{k \choose \pi} = \frac{k!}{\pi_1! \cdots \pi_n!}
\]
is the multinomial coefficient.

Substitution gives
\[
\begin{split}
0&= r^2 \psi_{rr} + Cr\psi_r - F(\psi) \\
&= \sum_{n=2}^{\infty} n(n-1) \psi_n r^n + C \sum_{n=1}^{\infty} n \psi_n r^n - \sum_{n=1}^{\infty} a_n r^n \\
&= (C \psi_1 - a_1) r + \sum_{n=2}^{\infty} \left[(n^2 + (C-1)n)\psi_n - a_n\right] r^n \\
&= (C - F_1) \psi_1 r + \sum_{n=2}^{\infty} \left[(n^2 +(C-1)n)\psi_n - a_n\right] r^n \\
\end{split}
\]

If our \(F\) and \(C\) are such that \(C = F_1\) (for example when \(C = 1\) and \(F'(0) = 1\)), we may freely choose \(\psi_1\). Otherwise, \(\psi_1 = 0\). Either way, we then seek to recursively solve
\[
(n^2 + (C-1)n)\psi_n - a_n = 0
\]
for \(\psi_n\). Note that in \(a_n\), from the definitions of \(S_{n,k}\) and \(P_{n,k}\) (namely that \(\sum_{i=1}^n \pi_i = k\) and \(\sum_{i=1}^n i \psi_i = n\)) the appearance of \(\psi_n\) is to first order and only in \(S_{n,1} = F_1 \psi_n\). This is just another way of saying that among the degree \(n\) terms of \(F(\psi)\), there is precisely one term, \(F_1 \psi_n\) involving \(\psi_n\).

Thus we have,
\[
a_n = F_1 \psi_n + q_n(\psi_1, \dots, \psi_{n-1})
\]
where
\[
q_n = \sum_{k=2}^n F_k \sum_{\pi \in P_{n,k}} {k \choose \pi} \psi_1^{\pi_1} \cdots \psi_k^{\pi_k}
\]
is a degree \(n\) polynomial in \(n-1\) variables with \(q_n(z_1, \cdots, z_{n-1}) = 0\). Therefore,
\[
\left[n^2 + (C-1) n - F_1\right] \psi_n =  q(\psi_1, \dots, \psi_{n-1}).
\]

Thus we have the recurrence relation,
\begin{equation}
\label{eq:recurrence}
p(n) \psi_n = q_n (\psi_1, \dots, \psi_{n-1})
\end{equation}
where \(p\) is the characteristic polynomial from equation \eqref{eq:char_poly} and in the case \(n=1\), \(p\) reduces to \(p(1) = C - F_1\) so that the recurrence relation \eqref{eq:recurrence} holds for \(n=1\) by defining \(q_1 = 0\).

If \(p(n) \ne 0\) for all natural numbers, we may choose \(\psi_1\) freely and solve for the remaining coefficients \(\psi_n\), \(n > 1\). In this situation, by choosing \(\psi = \lambda(t)\) appropriately and such that \(\lambda(t) \to \infty\) in finite time, we are able to construct a sub-solution with \(C^1\) blow up.

More generally, we have the following:

\begin{lemma}
\label{lem:heuristic_asymptotics}
Suppose \(n_0\) is a positive, integer root with \(p(n_0) = 0\). Then
\[
\psi(r) = \psi_0(r^{n_0})
\]
for some analytic function \(\psi_0\) with \(\psi_0(0) = 0\) and arbitrary \(\psi_0'(0)\). That is we may choose \(\psi_{n_0}\) arbitrarily, and \(\psi_n = 0\) for \(n\) not divisible by \(n_0\).
\end{lemma}

\begin{proof}
To see this, first observe that \(n_0 = 1\) if and only if
\[
1^2 + (C - 1) \cdot 1 - F_1 = 0 \Leftrightarrow C = F_1
\]
in which case
\[
p(n) = n^2 + (C - 1) n - C = (n-1)(n + C)
\]
and the other root is \(-C = -F_1 < 0\). That is \(p(n) \ne 0\) for all \(n > 1\) and we may solve the recurrence relation \eqref{eq:recurrence} uniquely for \(\psi_n, n > 1\) given any \(\psi_1\).

For \(n_0 > 1\) recall that we already observed that there is precisely one positive root of \(p\) and hence \(p(n) \ne 0\) for \(n \ne n_0\). Then we have \(p(1) \ne 0\) and recalling that we defined \(q_1 = 0\), the recurrence relation \eqref{eq:recurrence} is satisfied only if \(\psi_1 = 0\). But then observe that \(q_2 = q_1(\psi_1) = q_1(0) = 0\) and so
\[
p(2) \psi_2 = q_2 = 0
\]
and either \(p(2) = 0\) with \(\psi_2\) arbitrary if \(n_0 = 2\) or \(\psi_2 = 0\) if \(n_0 > 2\). If \(n_0 > 3\), we find \(q_3 = q_3(\psi_1, \psi_2) = q_3(0, 0) = 0\) implies \(\psi_3 = 0\). Continuing by induction we find that \(\psi_n = 0\) for all \(n\) with \(1 < n < n_0\). For \(n_0\) we get
\[
0 = p(n_0) \psi_{n_0} = q_{n_0-1} = 0
\]
and we may freely choose \(\psi_{n_0}\). So we have \(\psi_n = 0\) for \(n = 1, \cdots, n_0 - 1\) and \(\psi_{n_0}\) is arbitrary. Now we claim this implies \(q_{n_0 + j} = 0\) for \(j = 1, \cdots, n_0 - 1\). To see this, we recall that
\[
q_{n_0 + j} = \sum_{k=2}^{n_0 + j} F_k \sum_{\pi \in P_{n_0+j,k}} {k \choose \pi} \psi_1^{\pi_1} \cdots \psi_k^{\pi_k}
\]
with
\[
P_{n_0 + j,k} = \{\pi = (\pi_1, \dots, \pi_{n_0 + j}) : \sum_{i=1}^{n_0+j} \pi_i = k, \sum_{i=1}^{n_0+j} i \pi_i = n_0 + j\}.
\]
Consider the possible \(\pi\) with \(\pi_{n_0} \ne 0\). These must satisfy
\[
\sum_{i=1}^{n_0-1} i \pi_i + n_0 \pi_{n_0} + \sum_{i=n_0+1}^{n_0 + j} i\pi_i = n_0 + j
\]
which rearranges to
\[
\sum_{i=1}^{n_0-1} i \pi_i + \sum_{i=n_0+1}^{n_0 + j} i\pi_i = n_0(1 - \pi_{n_0}) + j
\]
If all the \(\pi_i\) with \(i \ne n_0\) vanish, then
\[
j = n_0(\pi_{n_0} - 1).
\]
But this is impossible since \(1 \leq j < n_0\) and \(\pi_{n_0} \geq 0\) is an integer.

Therefore, any \(\pi\) with and entry \(\pi_{n_0} \ne 0\) must also have \(\pi_i \ne 0\) for some \(i \ne n_0\). For \(j = 1\) we have \(q_{n_0 + 1}\) is a polynomial in \(\psi_i\) for \(1 \leq i \leq n_0\). But \(\psi_i = 0\) for \(1 \leq i \leq n_0 -1\) and every term of \(q_{n_0 + 1}\) has a term \(\psi_i^{\pi_i}\) with \(i \ne n_0\) and thus \(q_{n_0 + j}\) vanishes. Now continue by induction (e.g. every term of \(q_{n_0 + 2}\) has \(\psi_i^{\pi_i}\) with \(1 \leq i \leq n_0 - 1\) or \(i = n_0 + 1\) and hence vanishes).

Finally, a further induction argument may be used to show that \(\psi_{m n_0 +j} = 0\) for any natural number \(m\) and any \(j\) with \(1 \leq j \leq n_0 - 1\). The necessary ingredient is to show that any term of \(q_{m n_0 + j}\) involving \(\psi_{m n_0}^{\pi_{m n_0}}\) with \(\pi_{m n_0} \ne 0\) necessarily also includes terms of the form \(\psi_i^{\pi_i}\) with \(i\) not divisible by \(n_0\). This follows in the same way as before: we have
\[
\sum_{i=1}^m i n_0 \pi_{i n_0} + \sum_{i=1}^m \sum_{j=1}^{n_0 -1} (i n_0 + j) \pi_{i n_0 + j} = m n_0 + j
\]
rearranges to
\[
\sum_{i=1}^m \sum_{j=1}^{n_0 -1} (i n_0 + j) \pi_{i n_0 + j} = n_0\left(m - \sum_{i=1}^m i \pi_{i n_0}\right) + j
\]
and if the left hand side vanishes we have
\[
j = n_0\left(\sum_{i=1}^m i \pi_{i n_0} - m\right)
\]
but \(1 \leq j \leq n_0 - 1\) implies \(j\) is not divisible by \(n_0\) a contradiction. Thus at least one \(\pi_{im + j} \ne 0\) and we can repeat the same induction over \(j\).
\end{proof}

\subsubsection*{Example: Critical Dimension}

It's particularly instructive to consider the critical cases of two-dimensional harmonic map heat flow and four-dimensional Yang-Mills flow. In both cases \(C = 1\) (whereas in higher dimension \(C > 1\)). Thus in the critical dimension,
\[
p(\beta) = \beta^2 - A
\]
hence \(\beta_0 = \sqrt{A}\).

For the degree \(\ell\)-harmonic map heat flow, \(\beta_0 = \ell\) while for the Yang-Mills flow, \(\beta_0 = 2\). Thus for Yang-Mills and higher degree harmonic map heat flow (\(\ell > 1\)), we have \(\beta_0 > 1\) and no \(C^1\) blow up occurs. However for, \(\ell \leq 2\) and Yang-Mills we do expect \(C^2\) blow up. For \(\ell > 2\), there is no \(C^2\) blow up.

\subsection{Blow up at the origin}

With our heuristic motivation to hand, we move on to show that blow up, if it occurs at all, must occur at the origin. This is a straight forward application of standard estimates \cite[Theorem 10.1]{Ladyzhenskaja:/1967} but we record the details in the following lemma for convenience.

\begin{lemma}
\label{lem:apriori_bounds}
Let \(\psi\) be a solution of \eqref{eq:pde} with initial data \(\psi_0\) satisfying
\[
\|\psi_0\|_{C^{2,\alpha}([0, 1])} < \infty,
\]
and
\[
\|\psi\|_{L^{\infty} ([0, 1] \times [0, \tau])} < \infty
\]
If there exists a \(\rho \in (0, 1]\) and a \(\tau \in (0, T)\) such that
\[
\|\psi\|_{C^{2,\alpha}([0, \rho] \times [0, \tau])} < \infty,
\]
then
\[
\|\psi\|_{C^{2,\alpha}([0, 1] \times [0, \tau])} < \infty.
\]
In particular, if the maximal existence time \(T < \infty\), then
\[
\lim_{\tau\to T} \inf_{\rho \in (0, 1)} \{\|\psi\|_{C^2([0, \rho] \times [0, \tau])}\} = \infty.
\]
\end{lemma}

\begin{proof}
We need to show
\[
\|\psi\|_{C^{2,\alpha}([\rho, 1] \times [0, \tau])} < \infty,
\]
under the hypothesis
\[
\|\psi\|_{C^{2,\alpha}([0, \rho] \times [0, \tau])} < \infty.
\]

We use the notation of \cite[Theorem 10.1]{Ladyzhenskaja:/1967}. Define the linear operator
\[
\mathcal{L} \psi = \partial_t \psi - \psi_{rr} - \frac{C}{r} \psi_r
\]
and
\[
f(r, t) = -\frac{1}{r^2} F(\psi(r, t))
\]
so that
\[
\mathcal{L} \psi = f.
\]

Let \(\Omega = (0, 1)\), \(\Omega' = (\rho, 1)\), and \(\Omega'' = (\rho/2, 1)\), \(S = \{0, 1\}\), \(S' = S'' = \{1\}\). \cite[Theorem 10.1]{Ladyzhenskaja:/1967} gives constants \(c_1, c_2 > 0\) such that
\[
\begin{split}
\|\psi\|_{C^{2,\alpha}([\rho, 1] \times [0, \tau])} &\leq c_1 \left(\|\tfrac{1}{r^2} F(\psi)\|_{C^{0,\alpha}([\rho/2, 1] \times [0, \tau])} + \|\psi_0\|_{C^{2,\alpha}([\rho/2, 1])} + \|\psi\|_{C^{2,\alpha}(\{1\} \times [0, \tau])} \right) \\
&\quad + c_2 \|\psi\|_{C([\rho/2, 1] \times [0, \tau])}.
\end{split}
\]

In the hypothesis of the theorem, we assume \(\psi\) has finite \(C^{0,\alpha}\) norm, hence the last term is finite. Since \(F\) is also Lipschitz, the first term \(r^{-2} F(\psi(r, t))\) has finite \(C^{0,\alpha}\) norm on \(r \in [\rho, 1], t \in [0, \tau]\). The second term is the initial data \(\psi_0\) which is assumed to have finite \(C^{2,\alpha}\) norm and finally the third term - the boundary term \(\psi|_{r=1}\) - is constant hence also has finite norm.
\end{proof}

\subsection{Stationary solutions}

\textbf{The solutions here are \(C^{1,\alpha}\) weak solutions. Everything should work out just fine, but at the moment I've written them as classic solutions, i.e. as \(C^2\) functions. Later I'll fix it up and for now the computations are formal.}

In this section we study stationary solutions with suitable boundary data with a view to creating by sub- and super-solutions in the next section. Our stationary solutions satisfy
\begin{equation}
\label{eq:stationary}
\xi_{rr} + \frac{1}{r} \xi_r - \frac{1}{r^2} F(\xi)  = 0.
\end{equation}
Writing \(u = \xi, v = \xi_r\) this is equivalent to the system
\[
\begin{pmatrix}
u' \\
v'
\end{pmatrix}
=
\begin{pmatrix}
v \\
\frac{1}{r^2} F(u) - \frac{C}{r} v
\end{pmatrix}.
\]
For each fixed \(r > 0\), the right hand side is a Lipschitz function of \((u,v)\) for \(r>0\) since \(F\) is Lipschitz, but not uniformly in \(r\). The right hand side is also continuous with respect to \(r\) for \(r > 0\) but only measureable for \(r \in [0, 1]\). The Cartheodary theorem guarantees the existence of an absolutely continuous solution \((u, v)\) on \([0, \rho_1)\) for some  \(\rho_1 > 0\) given any initial condition \((u_0, v_0)\) at \(r=0\).

Any such solution \(\xi\) extends uniquely to \(r \in [0, 1]\) by choosing any \(\rho_2 \in (0, \rho_1)\) and solving the uniformly elliptic problem \eqref{eq:stationary} subject to the boundary conditions \(\psi(\rho_2) = \xi(\rho_2)\), \(\psi(1) = \psi_1\) on \([\rho_1, 1]\). Recall \(\psi_1\) is any real number occurring as the boundary data at \(r=1\) in \eqref{eq:pde}. The solution so obtained is uniformly \(C^{1,\alpha}\) on any interval \([\rho_2, 1]\) where \(\rho_2 > 0\).

Imposing the initial condition \(\psi(0) = 0\) gives a one-parameter family of solutions, parametrised by \(\psi'(0)\). Note that the Cartheodary theorem doesn't guarantee uniqueness but for our purposes, existence is sufficient.

\subsubsection*{Stationary solutions for \(C=1\)}

Let us first consider the case \(C = 1\) which is satisfied in critical dimension. Throughout this section we assume slightly more regularity than Lipschitz, namely that \(F\) is \(C^{1,\gamma}\).  We make the ansatz,
\[
\xi(r) = \xi_0(r^{\beta})
\]
for some \(C^{1,\alpha}\) function \(\xi_0\) with \(\xi_0(0) = 0\) and a real number \(\beta > 0\).

Then
\[
\begin{split}
\xi_{rr} + \frac{1}{r} \xi_r - \frac{1}{r^2} F(\xi) &= \beta^2 r^{2(\beta-1)} \xi_0'' + \beta (\beta - 1) r^{\beta-2} \xi_0' + \beta \xi_0' r^{\beta-2} - \frac{1}{r^2} F(\xi_0) \\
&= \beta^2 r^{-2} \left[(r^{\beta})^2 \xi_0'' + r^{\beta} \xi_0' + - \beta^{-2}F(\xi_0)\right].
\end{split}
\]

Writing \(y = r^{\beta}\) for the argument to \(\xi_0\), we see that \(\xi\) is a stationary solution satisfying \(\xi(0) = 0\) if and only if
\begin{equation}
\label{eq:stationary_ansatz}
\begin{cases}
y^2 \xi_0'' + y \xi_0' - \beta^{-2} F(\xi_0) &= 0 \\
\xi_0(0) &= 0.
\end{cases}
\end{equation}

The equation for \(\xi_0\) is effectively the same equation we started with, and we can solve it for any \(\beta > 0\) as in the previous section. Thus we aim to normalise \(\xi_0\) to obtain a canonical form. This canonical form is \(\xi_0'(0) = \lambda \ne 0\) in which case since \(\xi_0\) is \(C^{1,\alpha}\) we have
\[
\xi_0(r) = \lambda y + \mathcal{O}(y^{1+\alpha}).
\]
Before solving equation \eqref{eq:stationary_ansatz}, let us observe that with our normalisation, we must have \(\beta = F'(0)\): we have
\[
\xi_0' = \lambda + \mathcal{O}y^{\alpha}
\]
and
\[
\xi_0'' = \mathcal{O}^{\alpha-1}.
\]
Likewise,
\[
F(z) = F'(0) z + \mathcal{O}(z^{1+\gamma}).
\]
Substitution into equation \eqref{eq:stationary_ansatz} produces,
\[
\begin{split}
0 &= y^2 \mathcal{O}(y^{\alpha-1}) + y \left(\lambda + \mathcal{O}(y^{1+\alpha})\right) - \beta^{-2} F(\lambda y + \mathcal{O} (y^{1 + \alpha})) \\
&= \lambda(1 - \beta^{-2} F'(0)) y + \mathcal{O}(y^{1+\alpha})
\end{split}
\]
and hence \(1 = \beta^{-2} F'(0)\).

Now we can solve equation \eqref{eq:stationary_ansatz} explicitly (in terms of \(F\)) by changing variables, \(y = e^w\) and letting \(\zeta(w) = \xi_0(e^w)\), whence our problem becomes
\[
\begin{cases}
\zeta'' &= e^{2w} \xi_0'' + e^w \xi_0' = \beta^{-2} F(\zeta) \\
\zeta(-\infty) &= 0.
\end{cases}
\]
Multiplying by \(\zeta'\) this becomes
\[
[(\zeta')^2]' = 2 \beta^{-2} F(\zeta) \zeta'.
\]
Integrating gives
\[
(\zeta')^2 = G(\zeta) + C_1, \quad \text{where} \quad G'(Z) = 2\beta^{-2} F(Z).
\]
By adjusting \(C_1\) if necessary, we may assume that \(G(0) = 0\).

\textbf{Joe claims \(C_1 = 0\) follows from the initial condition, but I don't see why}

Thus we must solve
\[
\frac{\zeta'}{\sqrt{G(\zeta)}} = 1
\]
and hence our solution satisfies
\[
H (\zeta(w)) = w + \lambda, \quad \text{where} \quad H'(Z) = \frac{1}{\sqrt{G(Z)}}.
\]
for an arbitrary constant \(\lambda\).

In terms of \(\xi\),
\[
H (\xi(r, t)) = H (\xi_0(r^{\beta})) = \ln r^{\beta} + \lambda,
\]
and if \(H\) is invertible,
\[
\xi(r, t) = H^{-1} (\ln r^{\beta} + \lambda).
\]

\subsubsection*{Example: Harmonic map heat flow in \(\S^2\)}

Here we have
\[
F(Z) = \frac{\ell^2}{2} \sin(2Z)
\]
so that
\[
G(Z) = \frac{\ell^2}{\beta^2} \sin^2(Z)
\]
and
\[
H(Z) = \frac{1}{2} \ln \left(\frac{\cos(Z) - 1}{\cos(Z) + 1}\right)
\]

\subsubsection*{Example: Harmonic map heat flow in warped products}

Here \(F = \ell^2 uu' = \tfrac{\ell^2}{2} (u^2)'\) and so
\[
G = \frac{\ell^2}{\beta^2} u^2.
\]
\subsubsection*{Stationary solutions for arbitrary \(C\)}


\textbf{I guess we also want to prescribe \(\psi(1)\) so work out what this does to \(\psi'(0)\).}
\subsection{Barriers}

The argument in this section is based on constructing a one-parameter family of stationary-solutions from the stationary solutions constructed in the last section.

\end{document}
